\documentclass[colorlinks,linkcolor=teal,a4paper,11pt]{moderncv}
	
\usepackage{graphicx}
\usepackage{amssymb}
\usepackage{amsmath}
\usepackage[utf8]{inputenc}
\usepackage{longtable}
\usepackage{xcolor}
\usepackage{xspace}
\newcommand{\rsquo}{{\tt\char'023}}

\moderncvstyle{banking}
\moderncvcolor{green}

\definecolor{color1}{rgb}{0.0, 0.6, 0.6}
\definecolor{mark_color}{rgb}{0.5, 0.5, 0.5}

%\usepackage{cmbright}

\usepackage[sfdefault,lf]{carlito}
\usepackage[T1]{fontenc}
\renewcommand*\oldstylenums[1]{\carlitoOsF #1}

\usepackage[top=1.5cm,bottom=2cm,left=2cm,right=2cm,bindingoffset=0cm]{geometry}
\setlength{\hintscolumnwidth}{3cm}
\usepackage{enumitem}
\setlist{nolistsep}

\makeatletter
\renewcommand*{\bibliographyitemlabel}{\@biblabel{\arabic{enumiv}}}
\makeatother

\newcommand{\mnras}{Monthly Notices of the Royal Astronomical Society\xspace}
\newcommand{\mnrasl}{Monthly Notices of the Royal Astronomical Society Letters\xspace}
\newcommand{\jcap}{Journal of Cosmology and Astroparticle Physics\xspace}
\newcommand{\prd}{Physical Review D\xspace}
\newcommand{\prdl}{Physical Review D Letters\xspace}
\newcommand{\prdrc}{Physical Review D Rapid Communications\xspace}
\newcommand{\prx}{Physical Review X\xspace}
\newcommand{\prl}{\textbf{Physical Review Letters}\xspace}
\newcommand{\prlplain}{{Physical Review Letters\xspace}}
\newcommand{\cqg}{Classical and Quantum Gravity\xspace}
\newcommand{\aap}{Astronomy \& Astrophysics\xspace}
\newcommand{\prr}{Physical Review Research\xspace}
\newcommand{\apj}{Astrophysical Journal\xspace}
\newcommand{\apjl}{Astrophysical Journal Letters\xspace}
\newcommand{\ajp}{American Journal of Physics\xspace}
\newcommand{\grg}{General Relativity and Gravitation\xspace}
\newcommand{\natastro}{Nature Astronomy\xspace}
\newcommand{\lrr}{Living Reviews in Relativity\xspace}


\long\def\suppress#1\endsuppress{%
  \begingroup%
    \tracinglostchars=0%
    \let\selectfont=\nullfont
    \nullfont #1\endgroup}

\fancypagestyle{headonly}{
\fancyfoot{}
\fancyfoot[r]{\textcolor{color1}{\thepage}}
\fancyhead{}
}

\newcommand{\mytitle}[1]{\title{#1\vspace{0.15cm}}}

\firstname{Riccardo}
\familyname{Buscicchio}

\extrainfo{\normalsize riccardo.buscicchio@unimib.it $\;\;\bullet\;\;$ \href{http://www.riccardobuscicchio.com}{www.riccardobuscicchio.com}  $\;\;\bullet\;\;$ \today}


\mytitle{Curriculum Vit\ae}

\begin{document}
\hypersetup{urlcolor=color1}
		
\pagestyle{headonly}

\makecvtitle

%

%
\cvitem{Email}{\href{mailto:riccardo.buscicchio@unimib.it}{riccardo.buscicchio@unimib.it}}
%

%\cvitem{Citizenship}{Italy, EU.}

\section{Incarichi accademici}

\cventry{2021 - 2024}{Assegnista di ricerca, Dipartimento di Fisica ``G.Occhialini''}{Universit\`{a} degli Studi di Milano-Bicocca}{Milano, Italia}{}{}
\vspace{-0.1cm}
\begin{tabular}{rcl}
&\hspace{0.4cm} &$\circ\;\;${\textit{Attività di ricerca}}: sviluppo del segmento a terra per l'analisi dati di LISA per\\
&\hspace{0.4cm} &\phantom{$\circ\;\;${\textit{Attività di ricerca}} } l'Agenzia Spaziale Italiana (Fase A).
\end{tabular}

\cventry{2024 - 2027}{Assegnista di ricerca, Dipartimento di Fisica ``G.Occhialini''}{Universit\`{a} degli Studi di Milano-Bicocca}{Milano, Italia}{}{}
\vspace{-0.1cm}
\begin{tabular}{rcl}
&\hspace{0.4cm} &$\circ\;\;${\textit{Attività di ricerca}}: sviluppo del segmento a terra per l'analisi dati di LISA per\\
& & \phantom{$\circ\;\;${\textit{Attività di ricerca}} } l'Agenzia Spaziale Italiana (Fase B).
\end{tabular}

\vspace{-0.2cm}
\section{Istruzione}

\cventry{2017-13/07/2022}{Ph.D., School of Physics \& Astronomy}{University of Birmingham}{Birmingham, Regno Unito}{}{}
\vspace{-0.1cm}
\begin{tabular}{rcl}
&\hspace{0.4cm} &$\circ\;\;${\textit{Supervisore}}: A.~Vecchio. La tesi ha prodotto 6 pubblicazioni a lista di autori breve.
\\
&\hspace{0.4cm} &$\circ\;\;${\textit{Titolo della tesi}}:
Topics in Bayesian population inference for Gravitational Wave Astronomy
\end{tabular}
\vspace{0.2cm}

%

\vspace{0.2cm}
\cventry{2013-2016}{Laurea Magistrale in Fisica Teorica}{\newline Universit\`{a} degli Studi di Pisa}{Pisa, Italia}{}{}
\vspace{-0.1cm}
\begin{tabular}{rcl}
%
&\hspace{0.4cm} &$\circ\;\;${\textit{Supervisore}}: G.~Cella. La tesi ha prodotto una pubblicazione a lista di autori breve.\\
&\hspace{0.4cm} &$\circ\;\;${\textit{Titolo della tesi}}: 
An improved detector for non-gaussian stochastic background of gravitational waves.
\end{tabular}
\vspace{0.2cm}

%

\vspace{0.2cm}
\cventry{Giu-Sett 2013}{Programma di internship INFN-NSF}{Columbia University}{New York NY, Stati Uniti}{}{}
\vspace{-0.1cm}
\begin{tabular}{rcl}
&\hspace{0.4cm} &$\circ\;\;${\textit{Supervisori}}: S.~Marka, I.~Bartos.
\end{tabular}
\vspace{0.2cm}

%

%


%

\section{Indicatori bibliometrici}

\cvitem{}{\begin{tabular}{rcl}
\textcolor{mark_color}{\textbf{Pubblicazioni}}: & \hspace{0.3cm} & \\
&\textbf{27\, } & pubblicazioni short-author in riviste internazionali peer-reviewed\\
& & (di cui \textbf{7}\, articoli a primo autore e \textbf{5}\, di studenti supervisionati).\\
&\textbf{13} & articoli di collaborazione con contributo significativo in riviste internazionali peer-reviewed\\
&\textbf{47} & articoli di collaborazione totali, in riviste internazionali peer-reviewed\\
&\textbf{6}& \, articoli in fase preprint,\\
&\textbf{2}& \, altre pubblicazioni (tesi di dottorato, white papers, reviews)
\end{tabular} }
\textcolor{mark_color}{\textbf{Numero totale di citazioni}}: >14400.
\textcolor{mark_color}{\textbf{h-index}}: 23 (secondo record ADS e iNSPIRE).
\\
\textcolor{mark_color}{\textbf{Link a profili di citazione}}:
\href{https://ui.adsabs.harvard.edu/search/fq=%7B!type%3Daqp%20v%3D%24fq_doctype%7D&fq_doctype=(doctype%3A%22misc%22%20OR%20doctype%3A%22inproceedings%22%20OR%20doctype%3A%22article%22%20OR%20doctype%3A%22eprint%22)&q=%20author%3A%22Buscicchio%2C%20Riccardo%22&sort=citation_count%20desc%2C%20bibcode%20desc&p_=0}{\textsc{ADS}};
\href{https://inspirehep.net/literature?sort=mostrecent&size=25&page=1&q=author%3AR.Buscicchio&ui-citation-summary=true}{\textsc{iNSPIRE}};
\href{http://arxiv.org/a/buscicchio_r_1.html}{\textsc{arXiv}};
\href{https://orcid.org/0000-0002-7387-6754}{\textsc{orcid}}.

\textbf{Lista completa delle pubblicazioni} disponibile 
%
all'indirizzo \\
\href{http://www.riccardobuscicchio.com/publications}{www.riccardobuscicchio.com/publications}.

%\cvitem{}{\begin{tabular}{rcl}
\textcolor{mark_color}{\textbf{Seminari}}: &\hspace{0.3cm} &
\textbf{29} seminari a conferenze,
\textbf{10} seminari dipartimentali,
\\ & &
\end{tabular} }

\textbf{Lista completa dei seminari} disponibile
%
all'indirizzo \\
\href{http://www.riccardobuscicchio.com/talks}{www.riccardobuscicchio.com/talks}.

\section{Codici e datasets pubblici}

\begin{tabular}{@{\hskip 0.4cm}l@{\hskip 0.4in}c@{\hskip 0.1in}c@{\hskip 0.1in}l@{\hskip 0.1in}c}
\textbf{\textcolor{black}{Titolo}} & \textbf{\textcolor{black}{Codice}}& \textbf{\textcolor{black}{Dataset}} & \textbf{\textcolor{black}{Zenodo DOI}} & \textbf{\textcolor{black}{Pubblico}}\\
$\circ\;$ Hypertriangulation Map & \checkmark &  & \href{https://zenodo.org/record/13897708}{10.5281/zenodo.13897708} & \checkmark\\
$\circ\;$ Bayesian PowerLaw Sensitivity  & \checkmark& \checkmark & \href{https://zenodo.org/record/14384633}{10.5281/zenodo.14384633} & \checkmark\\
$\circ\;$ Milky Way Satellites & & \checkmark& \href{https://zenodo.org/record/3668904}{10.5281/zenodo.3668904} & \checkmark\\
$\circ\;$ LISA stellar BBH catalogues and samples  & \checkmark& \checkmark & \href{https://zenodo.org/record/14426778}{10.5281/zenodo.14426778} &\checkmark\\
$\circ\;$ LISA MBHB catalogues and samples& & \checkmark & \href{https://zenodo.org/record/13787674}{10.5281/zenodo.13787674} &\checkmark\\
$\circ\;$ Chirp Gravitational Wave Alerts & \checkmark &  & \href{https://zenodo.org/record/3525063}{10.5281/zenodo.3525063} &\checkmark\\
%$\circ\;$ FIGARO& C & \href{https://zenodo.org/record/6515965}{10.5281/zenodo.6515965} &\checkmark\\
\end{tabular}

\section{Grant, Premi \& Riconoscimenti}

\textbf{\textcolor{black}{Premi Accademici:}}
\vspace{0.1cm}

\cvitemwithcomment{}{\hspace{0.4cm}$\circ\;$ 
{Braccini PhD Thesis Prize}, Menzione d'onore della Gravitational Wave International Committee.}{2021}
\vspace{-0.1cm}
\cvitemwithcomment{}{\hspace{0.4cm}$\circ\;$ 
{Michael Penston PhD Thesis Prize}, Secondo premio della Royal Astronomical Society.}{2021}
\vspace{-0.1cm}

\vspace{0.2cm}

\textbf{\textcolor{black}{Grants:}}
\vspace{0.1cm}


\cvitemwithcomment{}{\hspace{0.4cm}$\circ\;$ 
{EuroHPC PRACE ``LISA-FIT'' proposal}, 100k CPUh su Leonardo BOOSTER}{2023}
\vspace{-0.1cm}

\cvitemwithcomment{}{\hspace{0.4cm}$\circ\;$ 
{Google Cloud for Researchers}, 4kEUR Google Cloud Research Credits}{2023}
\vspace{-0.1cm}

\cvitemwithcomment{}{\hspace{0.4cm}$\circ\;$ 
{CINECA ISCRA Type-C project ``LISA-MilkyWay''}, 10 kCPUh presso il Centro Nazionale per HPC.}{2022}
\vspace{-0.1cm}

\cvitemwithcomment{}{\hspace{0.4cm}$\circ\;${Grant di viaggio}, Horizon 2020 AHEAD 2020 (High Energy Astrophysics)}{2021}
\vspace{-0.1cm}

\cvitemwithcomment{}{\hspace{0.4cm}$\circ\;${Grant di viaggio}, American Physical Society, DGRAV Student Travel Grant}{2020}
\vspace{-0.1cm}

\cvitemwithcomment{}{\hspace{0.4cm}$\circ\;${Grant di viaggio}, Institute of Physics Student Travel fund}{2019}
\vspace{-0.1cm}

\cvitemwithcomment{}{\hspace{0.4cm}$\circ\;${Grant di viaggio}, Royal Astronomical Society, Regno Unito.}{2018}
\vspace{-0.1cm}

\section{Supervisione}
Secondo la normativa nazionale vigente, in qualità di assegnista di ricerca non posso essere nominato supervisore ufficiale di studenti di ogni grado. Tuttavia, previa autorizzazione da parte del personale strutturato rilevante, ho supervisionato l'attività degli studenti e delle studentesse nelle percentuali indicate.

\vspace{0.2cm}
\textbf{\textcolor{black}{Co-supervisione studenti di Dottorato:}}
\vspace{0.1cm}
\\
%
\cvitemwithcomment{}{\hspace{0.4cm}$\circ\;$ F.~Nobili, Università dell'Insubria, 100\%}{2024-2027}
\vspace{-0.1cm}
%
\cvitemwithcomment{}{\hspace{0.4cm}$\circ\;$ A.~Spadaro, Università di Milano-Bicocca. 100\%}{2022-2025}
\vspace{-0.1cm}
%
\cvitemwithcomment{}{\hspace{0.4cm}$\circ\;$ F.~Pozzoli, Università dell'Insubria. 100\%}{2022-2025}
\vspace{-0.1cm}

\vspace{0.2cm}
\textbf{\textcolor{black}{Co-supervisione studenti di Laurea magistrale:}}
\vspace{0.1cm}

\cvitemwithcomment{}{\hspace{0.4cm}$\circ\;$ L.~Viganò, Università di Milano-Bicocca, Tesi magistrale. 100\%}{2024-2025}
\vspace{-0.1cm}
%
\cvitemwithcomment{}{\hspace{0.4cm}$\circ\;$ M.~Bellotti, Università di Milano-Bicocca, Tesi magistrale. 100\%}{2024-2025}
\vspace{-0.1cm}
%
\cvitemwithcomment{}{\hspace{0.4cm}$\circ\;$ D.~Chirico, Università di Milano-Bicocca, Tesi magistrale. 100\%}{2023-2024}
\vspace{-0.1cm}
%
\cvitemwithcomment{}{\hspace{0.4cm}$\circ\;$ S.~Corbo, Politecnico di Milano, Tesi magistrale. 100\%}{2023-2024}
\vspace{-0.1cm}
%
\cvitemwithcomment{}{\hspace{0.4cm}$\circ\;$ R.~Rosso, Università di Pisa, Tesi magistrale. 80\%}{2023-2024}
\vspace{-0.1cm}
%
\cvitemwithcomment{}{\hspace{0.4cm}$\circ\;$ G.~Astorino, Università di Pisa, Tesi magistrale. 80\%}{2023-2024}
\vspace{-0.1cm}
%
\cvitemwithcomment{}{\hspace{0.4cm}$\circ\;$ M.~Piarulli, Università di Milano-Bicocca, Tesi magistrale. 100\%}{2022-2023}
\vspace{-0.1cm}
\hspace{0.4cm}$\phantom{\circ}\;$(ora studente di dottorato presso Univ. di Tolosa, Francia)
\vspace{0.1cm}

%
\cvitemwithcomment{}{\hspace{0.4cm}$\circ\;$ A.~Spadaro, Università di Milano-Bicocca, Tesi magistrale. 100\%}{2021-2022}
\vspace{-0.1cm}
\hspace{0.4cm}$\phantom{\circ}\;$(ora studentessa di dottorato presso Università di Milano-Bicocca, Italia)
\vspace{0.1cm}

%
\cvitemwithcomment{}{\hspace{0.4cm}$\circ\;$ A.~Carzaniga, Università di Milano-Bicocca, Tesi magistrale. 100\%}{2021-2022}
\vspace{-0.1cm}
%
\cvitemwithcomment{}{\hspace{0.4cm}$\circ\;$ A.~Geminardi, Università di Milano-Bicocca, Tesi magistrale. 100\%}{2021-2022}
\vspace{-0.1cm}
\hspace{0.4cm}$\phantom{\circ}\;$ (ora studente di dottorato presso Univ. di Pavia, Italia)
\vspace{0.1cm}

%
\cvitemwithcomment{}{\hspace{0.4cm}$\circ\;$ E.~Finch, Università di Birmingham, Tesi magistrale. 50\%}{2018}
\vspace{-0.1cm}
%
\cvitemwithcomment{}{\hspace{0.4cm}$\circ\;$ V.~Spasova, Università di Birmingham, Tesi magistrale. 50\%}{2018}
\vspace{-0.1cm}
%

\vspace{0.2cm}
\textbf{\textcolor{black}{Co-supervisione studenti di Laurea Triennale:}}
\vspace{0.1cm}

%
\cvitemwithcomment{}{\hspace{0.4cm}$\circ\;$ H.~P.~G.~Carabajo, Università di Milano-Bicocca, Tesi triennale. 100\%}{2023-2024}
\vspace{-0.1cm}
%

\section{Insegnamenti, assistenza alla didattica}

\vspace{0.2cm}
\textbf{\textcolor{black}{Insegnamenti:}}
\vspace{0.1cm}

\cvitemwithcomment{}{\hspace{0.4cm}$\circ\;$ \textbf{\textcolor{teal}{Current and future challenges in GW astronomy}}, Corso di Dottorato, Milano-Bicocca (Italia).}{2023}\vspace{-0.1cm}
\cvitemwithcomment{}{\phantom{\hspace{0.4cm}$\circ\;$} \textit{Contenuto}: Introduzione all'analisi dati per onde gravitazionali. Interferometri spaziali e terrestri.}{}\vspace{-0.1cm}
\cvitemwithcomment{}{\phantom{\hspace{0.4cm}$\circ\;$ Contenuto:}  Popolazioni di sorgenti attese: binarie di oggetti compatti, fondi stocastici.}{}\vspace{-0.1cm}

\cvitemwithcomment{}{\hspace{0.4cm}$\circ\;$ \textbf{\textcolor{teal}{Mathematical physics and gravity}} (MAF900), Corso di dottorato, Univ.~of Stavanger (Norvegia).}{2023}\vspace{-0.1cm}
\cvitemwithcomment{}{\phantom{\hspace{0.4cm}$\circ\;$} \textit{Contenuto}: Introduzione all'analisi dati per onde gravitazionali. Interferometri spaziali.}{}\vspace{-0.1cm}
\cvitemwithcomment{}{\phantom{\hspace{0.4cm}$\circ\;$ Contenuto:}  Popolazioni di sorgenti attese: binarie di oggetti compatti, fondi stocastici.}{}\vspace{-0.1cm}
\cvitemwithcomment{}{\phantom{\hspace{0.4cm}$\circ\;$ Contenuto:}  Modellistica di segnali: binarie di oggetti compatti galattici e extragalattici}{}\vspace{-0.1cm}
\cvitemwithcomment{}{\phantom{\hspace{0.4cm}$\circ\;$ Contenuto:}  fondi stocastici astrofisici e cosmologici.}{}\vspace{-0.1cm}
\cvitemwithcomment{}{\phantom{\hspace{0.4cm}$\circ\;$ Contenuto:}  Rivelazione e stima dei parametri di segnali: approcci frequentisti e Bayesiani.}{}\vspace{-0.1cm}
\cvitemwithcomment{}{\phantom{\hspace{0.4cm}$\circ\;$ Contenuto:}  Tecniche avanzate di campionamento stocastico.}{}\vspace{-0.1cm}

\cvitemwithcomment{}{\hspace{0.4cm}$\circ\;$ \textbf{\textcolor{teal}{Lezioni per il corso di Astrostatistica}} (F5802Q014/20), Laurea Magistrale in Astrofisica}{2022}\vspace{-0.1cm}
\cvitemwithcomment{}{\phantom{\hspace{0.4cm}$\circ\;$ } Univ. di Milano-Bicocca (Italia)}{}\vspace{-0.1cm}
\cvitemwithcomment{}{\phantom{\hspace{0.4cm}$\circ\;$ } \textit{Contenuto}:  Popolazioni di sorgenti attese: binarie di oggetti compatti, fondi stocastici.}{}\vspace{-0.1cm}
\cvitemwithcomment{}{\phantom{\hspace{0.4cm}$\circ\;$ Contenuto:} Introduzione all'analisi dati per onde gravitazionali. Interferometri spaziali.}{}\vspace{-0.1cm}
\cvitemwithcomment{}{\phantom{\hspace{0.4cm}$\circ\;$ Contenuto:}  Popolazioni di sorgenti attese: binarie di oggetti compatti, fondi stocastici.}{}\vspace{-0.1cm}
\cvitemwithcomment{}{\phantom{\hspace{0.4cm}$\circ\;$ Contenuto:}  Rivelazione e stima dei parametri di segnali: approcci frequentisti e Bayesiani.}{}\vspace{-0.1cm}

\vspace{0.2cm}
\cvitemwithcomment{}{\textbf{\textcolor{black}{Esercitatore}}}{\textbf{\textcolor{black}{Annualità}}}\vspace{-0.1cm}
\vspace{0.05cm}

\cvitemwithcomment{}{\hspace{0.4cm}$\circ\;$ \textbf{\textcolor{teal}{Python Computing Lab}}, Bachelor's degree in Physics, Univ. of Birmingham, Regno Unito}{2017-2021}\vspace{-0.1cm}
\cvitemwithcomment{}{\phantom{\hspace{0.4cm}$\circ\;$} \textit{Contenuto}: Programmazione in Python, simulazione e analisi di sistemi fisici in meccanica}{}\vspace{-0.1cm}
\cvitemwithcomment{}{\phantom{\hspace{0.4cm}$\circ\;$ Contenuto:}  classica e celeste, termodinamica, elettromagnetismo. Analisi dati.}{}\vspace{-0.1cm}

\cvitemwithcomment{}{\hspace{0.4cm}$\circ\;$ \textbf{\textcolor{teal}{Maths for physicists}}, Bachelor's degree in Physics, Univ. of Birmingham, Regno Unito}{2017-2019}\vspace{-0.1cm}
\cvitemwithcomment{}{\phantom{\hspace{0.4cm}$\circ\;$} \textit{Contenuto}: Algebra lineare, calcolo differenziale e integrale, equazioni differenziali}{}\vspace{-0.1cm}
\cvitemwithcomment{}{\phantom{\hspace{0.4cm}$\circ\;$ Contenuto:} teoria dei gruppi e delle rappresentazioni}{}\vspace{-0.1cm}

\cvitemwithcomment{}{\hspace{0.4cm}$\circ\;$ \textbf{\textcolor{teal}{Physics and communication skills}}, Master's degree in Physics, Univ. of Birmingham, Regno Unito.}{2019}\vspace{-0.1cm}
\cvitemwithcomment{}{\phantom{\hspace{0.4cm}$\circ\;$} \textit{Contenuto}: Basi di programmazione \LaTeX. Preparazione di report di laboratorio, pubblicazioni}{}\vspace{-0.1cm}
\cvitemwithcomment{}{\phantom{\hspace{0.4cm}$\circ\;$ Contenuto:}  seminari.}{}\vspace{-0.1cm}

\vspace{0.2cm}
%\newpage
\section{Responsabilità in collaborazioni internazionali, responsabilità editoriali e di ricerca}

\textbf{\textcolor{black}{Responsabilità collaborazioni internazionali}}
\vspace{0.1cm}

\cvitemwithcomment{}{\hspace{0.4cm}$\circ\;$ Co-chair della Coordination Unit L2D (Global Fit, ESA LISA Project Office)}{2024-2025}


\textbf{\textcolor{black}{Referee per riviste scientifiche}}
\vspace{0.1cm}

\begin{tabular}{@{\hskip 0.4cm}l@{\hskip 0.4in}l}
$\circ\;$ Physical Review Letters & $\circ\;$ Physical Review D \\
$\circ\;$ The Astrophysical Journal Letters  &  $\circ\;$  The Open Journal of Astrophysics  \\
$\circ\;$ Journal of Cosmology and Astroparticle Physics & $\circ\;$ NASA Technology Transfer Program \\
$\circ\;$ Monthly Notices of the Royal Astronomical Society& $\circ\;$ Classical and Quantum Gravity\\
$\circ\;$ Institute of Physics Trusted Reviewer Excellence program \\
\end{tabular}

\vspace{0.2cm}
\textbf{\textcolor{black}{Responsabilità editoriali}}
\vspace{0.1cm}

%\cvitemwithcomment{}{\hspace{0.4cm}$\circ\;$ MDPI Universe}{2024}\vspace{-0.1cm}
%\hspace{0.4cm}$\phantom{\circ}\;$ Special Issue \textit{``Challenges and Synergies with Future Gravitational Wave Observatories''.}\vspace{0.1cm}

\cvitemwithcomment{}{\hspace{0.4cm}$\circ\;$ Board editorial per il Lensing Working Group nella collaborazione LIGO,Virgo, KAGRA}{2023}\vspace{-0.1cm}

\cvitemwithcomment{}{\hspace{0.4cm}$\circ\;$ Co-editor dell'issue \text{``LISA data analysis''} per Living Review in Relativity}{2022-2023}\vspace{-0.1cm}

\vspace{0.2cm}
\textbf{\textcolor{black}{Organizzazione di conferenze e workshop}}
\vspace{0.1cm}

\cvitemwithcomment{}{\hspace{0.4cm}$\circ\;$ \textit{LISA Distributed Data Processing Center June Workshop}, Milano, Italia.}{2025}\vspace{-0.1cm}

\cvitemwithcomment{}{\hspace{0.4cm}$\circ\;$ \href{https://www.ifpu.it/}{IFPU focus week} on \href{https://sites.google.com/unimib.it/gwemerge/}{``\textit{Emerging methods in GW population inference}``}, Trieste, Italia.}{2024}\vspace{-0.1cm}

\cvitemwithcomment{}{\hspace{0.4cm}$\circ\;$ \textit{LISA Astrophysics Working Group Conference}, Birmingham, Regno Unito.}{2022}\vspace{-0.1cm}

\cvitemwithcomment{}{\hspace{0.4cm}$\circ\;$ \textit{Gravitational-wave populations: what's next?}, Milano, Italia.}{2023}\vspace{-0.1cm}

\cvitemwithcomment{}{\hspace{0.4cm}$\circ\;$ \textit{Gravitational-wave Excellence Alliance Training (GrEAT) PhD school}, Birmingham, Regno Unito.}{2019}\vspace{-0.1cm}

\cvitemwithcomment{}{\hspace{0.4cm}$\circ\;$ \textit{Gravitational-wave Open Science Center First Open Data Workshop}, (online)}{2019}\vspace{-0.1cm}

%

\vspace{0.2cm}
\textbf{\textcolor{black}{Riconoscimenti, qualifiche e cariche accademiche}}
\vspace{0.1cm}


\cvitemwithcomment{}{\hspace{0.4cm}$\circ\;$  Abilitazione Scientifica Nazionale a Professore di Seconda Fascia}{2023}\vspace{-0.1cm}
\hspace{0.4cm}$\phantom{\circ}\;$ (Settore 02/C1, GSD 02/PHYS-05 - SSD PHYS-05/A).

\cvitemwithcomment{}{\hspace{0.4cm}$\circ\;$ Abilitazione all'insegnamento accademico in Astrofisica (Sec.34).}{2023}\vspace{-0.1cm}
\hspace{0.4cm}$\phantom{\circ}\;$ Ministero francese dell'istruzione e della ricerca (qualificazione no.23234388826).

%

\vspace{0.2cm}
\textbf{\textcolor{black}{Affiliazioni accademiche}}
\vspace{0.1cm}

\cvitemwithcomment{}{\hspace{0.4cm}$\circ\;$ LISA Distributed Data Processing Center, full member.}{dal 2024}\vspace{-0.1cm}

\cvitemwithcomment{}{\hspace{0.4cm}$\circ\;$ LISA Consortium, core member.}{dal 2018}\vspace{-0.1cm}

\cvitemwithcomment{}{\hspace{0.4cm}$\circ\;$ Italian Center for Supercomputing (ICSC).}{dal 2021}\vspace{-0.1cm}

\cvitemwithcomment{}{\hspace{0.4cm}$\circ\;$ TEONGRAV National Initiative (Gravity Theory)}{dal 2021}\vspace{-0.1cm} 
\hspace{0.4cm}$\phantom{\circ}\;$ Italian National Institute for Nuclear Physics (INFN).

\cvitemwithcomment{}{\hspace{0.4cm}$\circ\;$ LIGO, Virgo, Kagra Collaboration, full member}{dal 2017}\vspace{-0.1cm}

\cvitemwithcomment{}{\hspace{0.4cm}$\circ\;$ Società italiana di relatività generale e fisica della gravitazione (SIGRAV)}{dal 2021}\vspace{-0.1cm}

\cvitemwithcomment{}{\hspace{0.4cm}$\circ\;$ Istituto Nazionale di Astrofisica (INAF)}{dal 2021}\vspace{-0.1cm}

\cvitemwithcomment{}{\hspace{0.4cm}$\circ\;$ American Physical Society (APS)}{}\vspace{-0.1cm}

\cvitemwithcomment{}{\hspace{0.4cm}$\circ\;$ Società italiana di fisica (SIF)}{2021}\vspace{-0.1cm}

\cvitemwithcomment{}{\hspace{0.4cm}$\circ\;$ Royal Astronomical Society (RAS) fellow.}{2018-2021}\vspace{-0.1cm}

%

%

%

\noindent
\begin{minipage}[t]{0.48\textwidth}
	\raggedright
	\textbf{Data:} \today
\end{minipage}
\hfill
\begin{minipage}[t]{0.48\textwidth}
	\raggedleft
	\textbf{Firma:} \rule{5cm}{0.4pt}
\end{minipage}

\end{document}
