\documentclass[colorlinks,linkcolor=teal,a4paper,11pt]{moderncv}
	
\usepackage{graphicx}
\usepackage{amssymb}
\usepackage{amsmath}
\usepackage[utf8]{inputenc}
\usepackage{longtable}
\usepackage{xcolor}
\usepackage{xspace}
\newcommand{\rsquo}{{\tt\char'023}}

\moderncvstyle{banking}
\moderncvcolor{green}

\definecolor{color1}{rgb}{0.0, 0.6, 0.6}
\definecolor{mark_color}{rgb}{0.5, 0.5, 0.5}

%\usepackage{cmbright}

\usepackage[sfdefault,lf]{carlito}
\usepackage[T1]{fontenc}
\renewcommand*\oldstylenums[1]{\carlitoOsF #1}

\usepackage[top=1.5cm,bottom=2cm,left=2cm,right=2cm,bindingoffset=0cm]{geometry}
\setlength{\hintscolumnwidth}{3cm}
\usepackage{enumitem}
\setlist{nolistsep}

\makeatletter
\renewcommand*{\bibliographyitemlabel}{\@biblabel{\arabic{enumiv}}}
\makeatother

\newcommand{\mnras}{Monthly Notices of the Royal Astronomical Society\xspace}
\newcommand{\mnrasl}{Monthly Notices of the Royal Astronomical Society Letters\xspace}
\newcommand{\jcap}{Journal of Cosmology and Astroparticle Physics\xspace}
\newcommand{\prd}{Physical Review D\xspace}
\newcommand{\prdl}{Physical Review D Letters\xspace}
\newcommand{\prdrc}{Physical Review D Rapid Communications\xspace}
\newcommand{\prx}{Physical Review X\xspace}
\newcommand{\prl}{\textbf{Physical Review Letters}\xspace}
\newcommand{\prlplain}{{Physical Review Letters\xspace}}
\newcommand{\cqg}{Classical and Quantum Gravity\xspace}
\newcommand{\aap}{Astronomy \& Astrophysics\xspace}
\newcommand{\prr}{Physical Review Research\xspace}
\newcommand{\apj}{Astrophysical Journal\xspace}
\newcommand{\apjl}{Astrophysical Journal Letters\xspace}
\newcommand{\ajp}{American Journal of Physics\xspace}
\newcommand{\grg}{General Relativity and Gravitation\xspace}
\newcommand{\natastro}{Nature Astronomy\xspace}
\newcommand{\lrr}{Living Reviews in Relativity\xspace}


\long\def\suppress#1\endsuppress{%
  \begingroup%
    \tracinglostchars=0%
    \let\selectfont=\nullfont
    \nullfont #1\endgroup}

\fancypagestyle{headonly}{
\fancyfoot{}
\fancyfoot[r]{\textcolor{color1}{\thepage}}
\fancyhead{}
}

\newcommand{\mytitle}[1]{\title{#1\vspace{0.15cm}}}

\firstname{Riccardo}
\familyname{Buscicchio}

\extrainfo{\normalsize riccardo.buscicchio@unimib.it $\;\;\bullet\;\;$ \href{http://www.riccardobuscicchio.com}{www.riccardobuscicchio.com}  $\;\;\bullet\;\;$ \today}


\mytitle{Curriculum Vit\ae}

\begin{document}
\hypersetup{urlcolor=color1}
		
\pagestyle{headonly}

\makecvtitle

%mark_CVshort
\cvitem{}{\emph{\vspace{-1.5cm}\\
$\quad$ Relativistic astrophysicist, developing advanced data analysis and statistical frameworks for Bayesian and frequentist methods. Applications include space-mission modelling, signal detection and parameter estimation in gravitational-wave astronomy, population inference, stochastic background searches.
}}
%mark_CVshort

%mark_CVshort
\section{Contacts}

%mark_CVshort
\cvitem{Email}{\href{mailto:riccardo.buscicchio@unimib.it}{riccardo.buscicchio@unimib.it}}
%mark_CVshort
\cvitem{Address}{Universit\`{a} degli Studi di Milano-Bicocca, Piazza della Scienza 3, 20126 Milano, Italy.}
\cvitem{Website \& publications record}{
	\href{https://www.riccardobuscicchio.com/}{www.riccardobuscicchio.com} -- \href{https://arxiv.org/a/buscicchio_r_1.html}{\textsc{arXiv}} --
	\href{https://orcid.org/0000-0002-7387-6754}{\textsc{ORCID}}
	}
%mark_CVshort

%\cvitem{Citizenship}{Italy, EU.}

\section{Academic positions}

\cventry{2021 - current}{Postdoctoral scholar (Assegnista di ricerca), Department of Physics ``G.Occhialini''}{Universit\`{a} degli Studi di Milano-Bicocca}{Milan, Italy}{}{}
\vspace{-0.1cm}
\begin{tabular}{rcl}
&\hspace{0.4cm} &$\circ\;\;${\textit{Main activity}}: development of LISA data analysis ground-segment for the Italian Space Agency.
\end{tabular}

\section{Education}

\cventry{2017-2021}{Ph.D., School of Physics \& Astronomy}{University of Birmingham}{Birmingham, UK}{}{}
\vspace{-0.1cm}
\begin{tabular}{rcl}
&\hspace{0.4cm} &$\circ\;\;${\textit{Supervisor}}: A.~Vecchio. Thesis resulted in 6 short-author publications.
\\
&\hspace{0.4cm} &$\circ\;\;${\textit{Thesis Title}}:
Topics in Bayesian population inference for Gravitational Wave Astronomy
\end{tabular}
\vspace{0.2cm}

%mark_CVshort
This thesis explores a number of topics related to Bayesian inference in gravitational-wave astronomy. From hierarchical inference on population of stellar mass binary black hole mergers, to the development of an end-to-end parameter estimation routine for space-based interferometers. Other topics are investigated: population of binary white dwarfs in satellite galaxies of the Milky Way; constraints from stochastic background on lensing of gravitational waves from binary neutron star and binary black hole mergers; statistical techniques for simultaneous inference on multiple undistinguishable sources.
%mark_CVshort

\vspace{0.2cm}
\cventry{2013-2016}{Master's degree in Theoretical physics}{\newline Universit\`{a} degli Studi di Pisa}{Pisa, Italy}{}{}
\vspace{-0.1cm}
\begin{tabular}{rcl}
%mark_CVshort
&\hspace{0.4cm} &$\circ\;\;${\textit{Final degree grade}}: 110/110\\
%mark_CVshort
&\hspace{0.4cm} &$\circ\;\;${\textit{Supervisor}}: G.~Cella. Thesis resulted in one short-author publications.\\
&\hspace{0.4cm} &$\circ\;\;${\textit{Thesis title}}: An improved detector for non-gaussian stochastic background of gravitational waves.
\end{tabular}
\vspace{0.2cm}

%mark_CVshort
This thesis explored the idea of using functional formalism from stochastic processes and classical field theory to develop a new detection algorithm, with improved performance, for non-gaussian stochastic backgrounds of gravitational waves
%mark_CVshort

\vspace{0.2cm}
\cventry{Jun-Sept 2013}{INFN-NSF Summer Internship}{Columbia University}{New York NY, USA}{}{}
\vspace{-0.1cm}
\begin{tabular}{rcl}
&\hspace{0.4cm} &$\circ\;\;${\textit{Supervisor}}: S.~Marka, I.~Bartos.
\end{tabular}
\vspace{0.2cm}

%mark_CVshort
We estimated the contribution to noise level in second and third generation ground-based detectors due to primary and secondary cosmic ray showers impinging on the interferometer mirrors.
%mark_CVshort

%mark_CVshort
\vspace{0.2cm}
\cventry{2008-2012}{Bachelor's degree in Physics}{Universit\`{a} degli Studi di Pisa}{Pisa, Italy}{}{}
\vspace{-0.1cm}
\begin{tabular}{rcl}
&\hspace{0.4cm} &$\circ\;\;${\textit{Final degree grade}}: 109/110.\\
&\hspace{0.4cm} &$\circ\;\;${\textit{Thesis title}}: Template banks for gravitational wave detection: an application of Information Geometry.
\end{tabular}
\vspace{0.2cm}
%mark_CVshort


%mark_CVshort
This thesis explored the idea of using differential geometry formalism (as defined in the context of Information theory) to develop a template placing algorithm over source parameter space with non-trivial manifold structure.
%mark_CVshort

%mark_CVshort
\newpage{}
%mark_CVshort


\section{Metrics}

\cvitem{}{\begin{tabular}{rcl}
\textcolor{mark_color}{\textbf{Pubblicazioni}}: & \hspace{0.3cm} & \\
&\textbf{27\, } & pubblicazioni short-author in riviste internazionali peer-reviewed\\
& & (di cui \textbf{7}\, articoli a primo autore e \textbf{5}\, di studenti supervisionati).\\
&\textbf{13} & articoli di collaborazione con contributo significativo in riviste internazionali peer-reviewed\\
&\textbf{47} & articoli di collaborazione totali, in riviste internazionali peer-reviewed\\
&\textbf{6}& \, articoli in fase preprint,\\
&\textbf{2}& \, altre pubblicazioni (tesi di dottorato, white papers, reviews)
\end{tabular} }
\textcolor{mark_color}{\textbf{Numero totale di citazioni}}: >14400.
\textcolor{mark_color}{\textbf{h-index}}: 23 (secondo record ADS e iNSPIRE).
\\
\textcolor{mark_color}{\textbf{Link a profili di citazione}}:
\href{https://ui.adsabs.harvard.edu/search/fq=%7B!type%3Daqp%20v%3D%24fq_doctype%7D&fq_doctype=(doctype%3A%22misc%22%20OR%20doctype%3A%22inproceedings%22%20OR%20doctype%3A%22article%22%20OR%20doctype%3A%22eprint%22)&q=%20author%3A%22Buscicchio%2C%20Riccardo%22&sort=citation_count%20desc%2C%20bibcode%20desc&p_=0}{\textsc{ADS}};
\href{https://inspirehep.net/literature?sort=mostrecent&size=25&page=1&q=author%3AR.Buscicchio&ui-citation-summary=true}{\textsc{iNSPIRE}};
\href{http://arxiv.org/a/buscicchio_r_1.html}{\textsc{arXiv}};
\href{https://orcid.org/0000-0002-7387-6754}{\textsc{orcid}}.

\textbf{Full list of publications} available 
%mark_CVshort
below and
%mark_CVshort
at \href{http://www.riccardobuscicchio.com/publications}{\texttt{www.riccardobuscicchio.com/publications}}.

%\cvitem{}{\begin{tabular}{rcl}
\textcolor{mark_color}{\textbf{Seminari}}: &\hspace{0.3cm} &
\textbf{29} seminari a conferenze,
\textbf{10} seminari dipartimentali,
\\ & &
\end{tabular} }

\textbf{Full list of presentations} available
%mark_CVshort
below and
%mark_CVshort
at \href{http://www.riccardobuscicchio.com/talks}{\texttt{www.riccardobuscicchio.com/talks}}.

\section{Grants, Prizes, \& Awards}

\textbf{\textcolor{black}{Career prizes:}}
\vspace{0.1cm}

\cvitemwithcomment{}{\hspace{0.4cm}$\circ\;$ 
{Braccini PhD Thesis Prize}, Gravitational Wave International Committee honorable mention.}{2021}
\vspace{-0.1cm}
\cvitemwithcomment{}{\hspace{0.4cm}$\circ\;$ 
{Michael Penston PhD Thesis Prize}, Royal Astronomical Society runner-up prize.}{2021}
\vspace{-0.1cm}

%%%%%%%
\vspace{0.2cm}

\textbf{\textcolor{black}{Other funding:}}
\vspace{0.1cm}


\cvitemwithcomment{}{\hspace{0.4cm}$\circ\;$ 
{EuroHPC PRACE ``LISAS-FIT'' proposal}, 100k CPUh on Leonardo BOOSTER}{2023}
\vspace{-0.1cm}

\cvitemwithcomment{}{\hspace{0.4cm}$\circ\;$ 
{Google Cloud for Researchers}, 4kEUR Google Cloud Research Credits}{2023}
\vspace{-0.1cm}

\cvitemwithcomment{}{\hspace{0.4cm}$\circ\;$ 
{CINECA ISCRA Type-C project ``LISA-MW'' proposal}, 10k CPUh at the Italian National HPC center.}{2022}
\vspace{-0.1cm}

\cvitemwithcomment{}{\hspace{0.4cm}$\circ\;${Travel Grant}, Horizon 2020 AHEAD 2020 (High Energy Astrophysics)}{2021}
\vspace{-0.1cm}

\cvitemwithcomment{}{\hspace{0.4cm}$\circ\;${Travel Grant}, American Physical Society, DGRAV Student Travel Grant}{2020}
\vspace{-0.1cm}

\cvitemwithcomment{}{\hspace{0.4cm}$\circ\;${Travel Grant}, Institute of Physics Student Travel fund}{2019}
\vspace{-0.1cm}

\cvitemwithcomment{}{\hspace{0.4cm}$\circ\;${Travel Grant}, Royal Astronomical Society, UK.}{2018}
\vspace{-0.1cm}

\vspace{0.2cm}
\textbf{\textcolor{black}{Ph.D. student co-supervisor:}}
\vspace{0.1cm}
\\
\cvitemwithcomment{}{\hspace{0.4cm}$\circ\;$ A.~Spadaro, University of Milano-Bicocca.}{2022-2025}
\vspace{-0.1cm}
%
\cvitemwithcomment{}{\hspace{0.4cm}$\circ\;$ F.~Pozzoli, University of Insubria.}{2022-2025}
\vspace{-0.1cm}

\vspace{0.2cm}
\textbf{\textcolor{black}{MSc student mentoring:}}
\vspace{0.1cm}

\cvitemwithcomment{}{\hspace{0.4cm}$\circ\;$ M.~Piarulli, University of Milano-Bicocca, Master's thesis.}{2022-2023}
\vspace{-0.1cm}
\hspace{0.4cm}$\phantom{\circ}\;$(now PhD student at Univ. of Toulouse)
\vspace{0.1cm}

%
\cvitemwithcomment{}{\hspace{0.4cm}$\circ\;$ A.~Spadaro, University of Milano-Bicocca, Master's thesis.}{2021-2022}
\vspace{-0.1cm}
\hspace{0.4cm}$\phantom{\circ}\;$(now PhD student at Univ. of Milano-Bicocca)
\vspace{0.1cm}

%
\cvitemwithcomment{}{\hspace{0.4cm}$\circ\;$ A.~Carzaniga, University of Milano-Bicocca, Master's thesis.}{2021-2022}
\vspace{-0.1cm}
%
\cvitemwithcomment{}{\hspace{0.4cm}$\circ\;$ A.~Geminardi, University of Milano-Bicocca, Master's thesis.}{2021-2022}
\vspace{-0.1cm}
\hspace{0.4cm}$\phantom{\circ}\;$ (now PhD student at Univ. of Pavia)
\vspace{0.1cm}

%
\cvitemwithcomment{}{\hspace{0.4cm}$\circ\;$ E.~Finch, University of Birmingham, Year 4 project.}{2018}
\vspace{-0.1cm}
%
\cvitemwithcomment{}{\hspace{0.4cm}$\circ\;$ V.~Spasova, University of Birmingham, Year 4 project.}{2018}
\vspace{-0.1cm}
%

\vspace{0.2cm}
\textbf{\textcolor{black}{BSc student mentoring:}}
\vspace{0.1cm}

%
\cvitemwithcomment{}{\hspace{0.4cm}$\circ\;$ H.~P.~G.~Carabajo, University of Milano-Bicocca, Bachelor's thesis.}{2023-2024}
\vspace{-0.1cm}
%

\section{Teaching}

\vspace{0.2cm}
\textbf{\textcolor{black}{Taught classes:}}
\vspace{0.1cm}

\cvitemwithcomment{}{\hspace{0.4cm}$\circ\;$ Current and future challenges in GW astronomy, PhD course, Milano-Bicocca (IT).}{2023}\vspace{-0.1cm}

\cvitemwithcomment{}{\hspace{0.4cm}$\circ\;$ Mathematical physics and gravity (MAF900), Module 3, PhD course, Univ.~of Stavanger (NO).}{2023}\vspace{-0.1cm}

\cvitemwithcomment{}{\hspace{0.4cm}$\circ\;$ Contributed lecture to Astrostatistics (F5802Q014/20), Astrophysics MSc, Milano-Bicocca.}{2022}\vspace{-0.1cm}


\vspace{0.2cm}
\textbf{\textcolor{black}{Tutoring:}}
\vspace{0.05cm}

\cvitemwithcomment{}{\hspace{0.4cm}$\circ\;$ Year 2 Python Computing Lab, Physics BSc, Birmingham.}{2017-2021}\vspace{-0.1cm} 

\cvitemwithcomment{}{\hspace{0.4cm}$\circ\;$ Year 2 Maths for physicists, Physics BSc, Birmingham.}{2017-2019}\vspace{-0.1cm}

\cvitemwithcomment{}{\hspace{0.4cm}$\circ\;$ Year 4 Physics and communication skills, Physics BSc, Birmingham.}{2019}\vspace{-0.1cm}

\vspace{0.2cm}

\section{Academic service}

\textbf{\textcolor{black}{Journal referee}}
\vspace{0.1cm}

\begin{tabular}{@{\hskip 0.4cm}l@{\hskip 0.4in}l}
$\circ\;$ Physical Review Letters & $\circ\;$ Physics Letters D \\
$\circ\;$  The Astrophysical Journal Letters  &  $\circ\;$  The Open Journal of Astrophysics  \\
$\circ\;$  Journal of Cosmology and Astroparticle Physics & $\circ\;$ NASA Technology Transfer Program \\
$\circ\;$  Monthly Notices of the Royal Astronomical Society& \\
\end{tabular}

\vspace{0.2cm}
\textbf{\textcolor{black}{Editorial responsibilities}}
\vspace{0.1cm}

\cvitemwithcomment{}{\hspace{0.4cm}$\circ\;$ MDPI Universe}{2024}\vspace{-0.1cm}
\hspace{0.4cm}$\phantom{\circ}\;$ Special Issue \textit{``Challenges and Synergies with Future Gravitational Wave Observatories''.}\vspace{0.1cm}

\cvitemwithcomment{}{\hspace{0.4cm}$\circ\;$ Editorial board Lensing subgroup in the LIGO,Virgo, KAGRA collaboration}{2023}\vspace{-0.1cm}

\cvitemwithcomment{}{\hspace{0.4cm}$\circ\;$ Co-editor of a Living Review in relativity issue on \text{``LISA data analysis''} }{2022-2023}\vspace{-0.1cm}

\vspace{0.2cm}
\textbf{\textcolor{black}{Conference organizer (or committee member)}}
\vspace{0.1cm}

\cvitemwithcomment{}{\hspace{0.4cm}$\circ\;$ \href{https://www.ifpu.it/}{IFPU focus week} on \href{https://sites.google.com/unimib.it/gwemerge/}{``\textit{Emerging methods in GW population inference}``}, Trieste, IT.}{2024}\vspace{-0.1cm}

\cvitemwithcomment{}{\hspace{0.4cm}$\circ\;$ \textit{LISA Astrophysics Working Group Conference}, Birmingham, UK.}{2022}\vspace{-0.1cm}

\cvitemwithcomment{}{\hspace{0.4cm}$\circ\;$ \textit{Gravitational-wave populations: what's next?}, Milan, Italy.}{2023}\vspace{-0.1cm}

\cvitemwithcomment{}{\hspace{0.4cm}$\circ\;$ \textit{Gravitational-wave Excellence Alliance Training (GrEAT) PhD school}, Birmingham, UK.}{2019}\vspace{-0.1cm}

\cvitemwithcomment{}{\hspace{0.4cm}$\circ\;$ \textit{Gravitational-wave Open Science Center First Open Data Workshop}, Remote}{2019}\vspace{-0.1cm}

%mark_CVshort
\vspace{0.2cm}
\textbf{\textcolor{black}{Outreach \& public engagement}}
\vspace{0.1cm}

\cvitemwithcomment{}{\hspace{0.4cm}$\circ\;$ Development of illustrations and animations for LISA Consortium}{2023}\vspace{-0.1cm}

\cvitemwithcomment{}{\hspace{0.4cm}$\circ\;$ Development of illustrations and graphic content for LIGO Magazine}{2022-2023}\vspace{-0.1cm}

\cvitemwithcomment{}{\hspace{0.4cm}$\circ\;$ Development of visualisation interface and skymaps content for GW alerts}{2022-2023}\vspace{-0.1cm}
\hspace{0.4cm}$\phantom{\circ}\;$ web and smartphone app. \href{https://chirp.sr.bham.ac.uk/}{\texttt{https://chirp.sr.bham.ac.uk}}

\cvitemwithcomment{}{\hspace{0.4cm}$\circ\;$ Organization of biweekly public engagement events \textit{``Astronomy in the city''}, Birmingham, UK}{2017-2021}\vspace{-0.1cm}

\cvitemwithcomment{}{\hspace{0.4cm}$\circ\;$ Lectures to high-school students, Italy}{since 2021}\vspace{-0.1cm}
%mark_CVshort

%mark_CVshort
\vspace{0.2cm}
\textbf{\textcolor{black}{Professional recognition and service}}
\vspace{0.1cm}

\cvitemwithcomment{}{\hspace{0.4cm}$\circ\;$  Italian Habilitation  to Associate Professorship in Astrophysics (ASN 02/C1).}{2023}

\cvitemwithcomment{}{\hspace{0.4cm}$\circ\;$  Postdoc representative, Department of Physics, University of Milano-Bicocca}{2023-2025}\vspace{-0.1cm}

\cvitemwithcomment{}{\hspace{0.4cm}$\circ\;$ French Qualification for teaching in Higher Education in Astrophysics (Sec.34).}{2023}\vspace{-0.1cm}
\hspace{0.4cm}$\phantom{\circ}\;$ French Ministry of higher education and research (qualification no.23234388826).

\cvitemwithcomment{}{\hspace{0.4cm}$\circ\;$ Meetings organizer ``PhD meet and greet'', University of Birmingham}{2021}\vspace{-0.1cm}

\cvitemwithcomment{}{\hspace{0.4cm}$\circ\;$ LSC Academic Advisory Committee.}{2019-2021}\vspace{-0.1cm}
%mark_CVshort

\vspace{0.2cm}
\textbf{\textcolor{black}{Memberships}}
\vspace{0.1cm}

\cvitemwithcomment{}{\hspace{0.4cm}$\circ\;$ Italian Center for Supercomputing (ICSC).}{since 2021}\vspace{-0.1cm}

\cvitemwithcomment{}{\hspace{0.4cm}$\circ\;$ TEONGRAV National Initiative (Gravity Theory)}{since 2021}\vspace{-0.1cm} 
\hspace{0.4cm}$\phantom{\circ}\;$ Italian National Institute for Nuclear Physics (INFN).

\cvitemwithcomment{}{\hspace{0.4cm}$\circ\;$ LIGO, Virgo, Kagra Collaboration, full member.}{since 2017}\vspace{-0.1cm}

\cvitemwithcomment{}{\hspace{0.4cm}$\circ\;$ LISA Consortium, full member.}{since 2018}\vspace{-0.1cm}

\cvitemwithcomment{}{\hspace{0.4cm}$\circ\;$ Italian Society of General Relativity and Gravitational Physics (SIGRAV)}{since 2021}\vspace{-0.1cm}

\cvitemwithcomment{}{\hspace{0.4cm}$\circ\;$ Istituto Nazionale di Astrofisica (INAF)}{since 2021}\vspace{-0.1cm}

\cvitemwithcomment{}{\hspace{0.4cm}$\circ\;$ American Physical Society (APS), member.}{}\vspace{-0.1cm}

\cvitemwithcomment{}{\hspace{0.4cm}$\circ\;$ Italian Physical Society (SIF), member}{2021}\vspace{-0.1cm}

\cvitemwithcomment{}{\hspace{0.4cm}$\circ\;$ Royal Astronomical Society (RAS), fellow.}{2018-2021}\vspace{-0.1cm}

%mark_CVshort
\section{Skills}
\cvitem{Programming languages}{Python (advanced), Bash (advanced), Go, R, Stan, Julia, Mathematica, C, Qt5.}
\cvitem{Other scientific tools}{TensorFlow, LIGO lalsuite, \LaTeX, git, HPC tools, containerization, continous integration, cloud computing, website development.}
\cvitem{Languages}{English (fluent), Italian (native), French (basic)}
%mark_CVshort

%mark_CVshort
\section{Hobbies}
Swimming, running, rock climbing, photography. Sci-fi books, electronic music.
%mark_CVshort

%mark_CVshort
\pagebreak
\section{Full publication list}\vspace{0.2cm} 

\textcolor{color1}{\textbf{Submitted short-author and collaboration papers which I have substantially contributed to.:}}
\vspace{-0.5cm}

\cvitem{}{\small\hspace{-1cm}\begin{longtable}{rp{0.3cm}p{15.8cm}}
%
\textbf{5.} & & \textit{Functional inference on deviations from General Relativity.}
\newline{}
C. Pacilio, \textbf{R. Buscicchio}.
\newline{}
\href{https://arxiv.org/abs/2507.13454[gr-qc]}{arXiv:2507.13454[gr-qc].}
\vspace{0.09cm}\\
%
\textbf{4.} & & \textit{Comparing astrophysical models to gravitational-wave data in the observable space.}
\newline{}
A. Toubiana, D. Gerosa, M. Mould, S. Rinaldi, M. Arca Sedda, T. Bruel, \textbf{R. Buscicchio}, J. Gair, L. Paiella, F. Santoliquido, R. Tenorio, C. Ugolini.
\newline{}
\href{https://arxiv.org/abs/2507.13249[gr-qc]}{arXiv:2507.13249[gr-qc].}
\vspace{0.09cm}\\
%
\textbf{3.} & & \textit{Bahamas: BAyesian inference with HAmiltonian Montecarlo for Astrophysical Stochastic background.}
\newline{}
F. Pozzoli, \textbf{R. Buscicchio}, A. Klein, D. Chirico.
\newline{}
\href{https://arxiv.org/abs/2506.22542[astro-ph.IM]}{arXiv:2506.22542[astro-ph.IM].}
\vspace{0.09cm}\\
%
\textbf{2.} & & \textit{LISA Definition Study Report.}
\newline{}
M. Colpi, K. Danzmann, M. Hewitson, K. Holley-Bockelmann, et al. (incl. \textbf{R. Buscicchio}).
\newline{}
\href{https://arxiv.org/abs/2402.07571}{arXiv:2402.07571 [astro-ph.CO].}
\vspace{0.09cm}\\
%
\textbf{1.} & & \textit{The last three years: multiband gravitational-wave observations of stellar-mass binary black holes.}
\newline{}
A. Klein, G. Pratten, \textbf{R. Buscicchio}, P. Schmidt, C. J. Moore, E. Finch, A. Bonino, L. M. Thomas, N. Williams, D. Gerosa, S. McGee, M. Nicholl, A. Vecchio.
\newline{}
\href{https://arxiv.org/abs/2204.03423}{arXiv:2204.03423 [astro-ph.HE].}
\vspace{0.09cm}\\
%
\end{longtable} }
\textcolor{color1}{\textbf{Short-author papers in major peer-reviewed journals:}}
\vspace{-0.5cm}

\cvitem{}{\small\hspace{-1cm}\begin{longtable}{rp{0.3cm}p{15.8cm}}
%
\textbf{32.} & & \textit{Environmental effects in the LISA stochastic signal from stellar-mass black hole binaries.}
\newline{}
R. Chen, R. S. Chandramouli, F. Pozzoli, \textbf{R. Buscicchio}, E. Barausse.
\newline{}
\href{https://doi.org/10.1103/w61d-3jk5}{\prd 112, (2025), (in press)}. \href{https://arxiv.org/abs/2507.00694[gr-qc]}{arXiv:2507.00694[gr-qc].}
\vspace{0.09cm}\\
%
\textbf{31.} & & \textit{Variability in the massive black hole binary candidate SDSS J2320+0024: no evidence for periodic modulation.}
\newline{}
F. Rigamonti, L. Bertassi, \textbf{R. Buscicchio}, F. Cocchiararo, S. Covino, M. Dotti, A. Sesana, P. Severgnini.
\newline{}
\href{https://doi.org/10.1051/0004-6361/202555550}{\aap (2025), (in press)}. \href{https://arxiv.org/abs/2505.22706[astro-ph.GA]}{arXiv:2505.22706[astro-ph.GA].}
\vspace{0.09cm}\\
%
\textbf{30.} & & \textit{Is your stochastic signal really detectable?.}
\newline{}
F. Pozzoli, J. Gair, \textbf{R. Buscicchio}, L. Speri.
\newline{}
\href{https://doi.org/10.1103/22h4-tqh9}{\prd 112, (2025) 064035}. \href{https://arxiv.org/abs/2412.10468}{arXiv:2412.10468 [astro-ph.IM].}
\vspace{0.09cm}\\
%
\textbf{29.} & & \textit{A test for LISA foreground Gaussianity and stationarity. I. Galactic white-dwarf binaries.}
\newline{}
\textbf{R. Buscicchio}, A. Klein, V. Korol, F. Di Renzo, C.J. Moore, D. Gerosa, A. Carzaniga.
\newline{}
\href{https://doi.org/10.1140/epjc/s10052-025-14616-w}{\epjc 85, (2025) 887}. \href{https://arxiv.org/abs/2410.08263}{arXiv:2410.08263 [astro-ph.HE].}
\vspace{0.09cm}\\
%
\textbf{28.} & & \textit{Accelerating LISA inference with Gaussian processes.}
\newline{}
J. El Gammal, \textbf{R. Buscicchio}, G. Nardini, J. Torrado.
\newline{}
\href{https://doi.org/10.1103/c66v-rl3w}{\prd 112, (2025) 063010}. \href{https://arxiv.org/abs/2503.21871}{arXiv:2503.21871 [astro-ph.HE].}
\vspace{0.09cm}\\
%
\textbf{27.} & & \textit{Test for LISA foreground Gaussianity and stationarity: extreme mass-ratio inspirals.}
\newline{}
M. Piarulli, \textbf{R. Buscicchio}, F. Pozzoli, O. Burke, M. Bonetti, A. Sesana.
\newline{}
\href{https://doi.org/10.1103/nfn4-pgr5}{\prd 111, (2025) 103047}. \href{https://arxiv.org/abs/2410.08862}{arXiv:2410.08862 [astro-ph.HE].}
\vspace{0.09cm}\\
%
\textbf{26.} & & \textit{Cyclostationary signals in LISA: a practical application to Milky Way satellites.}
\newline{}
F. Pozzoli, \textbf{R. Buscicchio}, A. Klein, V. Korol, A. Sesana, F. Haardt.
\newline{}
\href{https://doi.org/10.1103/PhysRevD.111.063005}{\prd 111, (2025) 063005}. \href{https://arxiv.org/abs/2410.08274}{arXiv:2410.08274 [astro-ph.GA].}
\vspace{0.09cm}\\
%
\textbf{25.} & & \textit{Characterization of non-Gaussian stochastic signals with heavier-tailed likelihoods.}
\newline{}
N. Karnesis, A. Sasli, \textbf{R. Buscicchio}, N. Stergioulas.
\newline{}
\href{https://doi.org/10.1103/PhysRevD.111.022005}{\prd 111, (2025) 022005}. \href{https://arxiv.org/abs/2410.14354}{arXiv:2410.14354 [gr-qc].}
\vspace{0.09cm}\\
%
\textbf{24.} & & \textit{Stellar-mass black-hole binaries in LISA: characteristics and complementarity with current-generation interferometers.}
\newline{}
\textbf{R. Buscicchio}, J. Torrado, C. Caprini, G. Nardini, M. Pieroni, N. Karnesis, A. Sesana.
\newline{}
\href{https://doi.org/10.1088/1475-7516/2025/01/084}{\jcap 01 (2025) 084}. \href{https://arxiv.org/abs/2410.18171}{arXiv:2410.18171 [astro-ph.HE].}
\vspace{0.09cm}\\
%
\textbf{23.} & & \textit{Stars or gas? Constraining the hardening processes of massive black-hole binaries with LISA.}
\newline{}
A. Spadaro, \textbf{R. Buscicchio}, D. Izquierdo--Villalba, D. Gerosa, A. Klein, G. Pratten.
\newline{}
\href{https://doi.org/10.1103/PhysRevD.111.023004}{\prd 111, (2025) 023004}. \href{https://arxiv.org/abs/2409.13011}{arXiv:2409.13011 [astro-ph.HE].}
\vspace{0.09cm}\\
%
\textbf{22.} & & \textit{Partial alignment between jets and megamasers: coherent or selective accretion?.}
\newline{}
M. Dotti, \textbf{R. Buscicchio}, F. Bollati, R. Decarli, W. Del Pozzo, A. Franchini.
\newline{}
\href{https://doi.org/10.1051/0004-6361/202450112}{\aap 692 (2024) A233}. \href{https://arxiv.org/abs/2403.18002}{arXiv:2403.18002 [astro-ph.GA].}
\vspace{0.09cm}\\
%
\textbf{21.} & & \textit{Expected insights on type Ia supernovae from LISA's gravitational wave observations.}
\newline{}
V. Korol, \textbf{R. Buscicchio}, Ruediger Pakmor, Javier Morán-Fraile, Christopher J. Moore, Selma E. de Mink.
\newline{}
\href{https://www.aanda.org/articles/aa/full_html/2024/11/aa51380-24/aa51380-24.html}{\aap 691 (2024) A44}. \href{https://arxiv.org/abs/2407.03935}{arXiv:2407.03935 [astro-ph.HE].}
\vspace{0.09cm}\\
%
\textbf{20.} & & \textit{A weakly-parametric approach to stochastic background inference in LISA.}
\newline{}
F. Pozzoli, \textbf{R. Buscicchio}, C. J. Moore, A. Sesana, F. Haardt, A. Sesana.
\newline{}
\href{https://journals.aps.org/prd/abstract/10.1103/PhysRevD.109.083029}{\prd 109, (2024) 083029}. \href{https://arxiv.org/abs/2311.12111}{arXiv:2311.12111 [astro-ph.CO].}
\vspace{0.09cm}\\
%
\textbf{19.} & & \textit{A fast test for the identification and confirmation of massive black hole binary.}
\newline{}
M. Dotti, F. Rigamonti, S. Rinaldi, W. Del Pozzo, R. Decarli, \textbf{R. Buscicchio}.
\newline{}
\href{https://www.aanda.org/articles/aa/abs/2023/12/aa46916-23/aa46916-23.html}{\aap 680 (2023) A69}. \href{https://arxiv.org/abs/2310.06896}{arXiv:2310.06896 [astro-ph.HE].}
\vspace{0.09cm}\\
%
\textbf{18.} & & \textit{Glitch systematics on the observation of massive black-hole binaries with LISA.}
\newline{}
A. Spadaro, \textbf{R. Buscicchio}, D. Vetrugno, A. Klein, D. Gerosa, S. Vitale, R. Dolesi, W. J. Weber, M. Colpi.
\newline{}
\href{https://journals.aps.org/prd/abstract/10.1103/PhysRevD.108.123029}{\prd 108 (2023) 123029}. \href{https://arxiv.org/abs/2306.03923}{arXiv:2306.03923 [gr-qc].}
\vspace{0.09cm}\\
%
\textbf{17.} & & \textit{Implications of pulsar timing array observations for LISA detections of massive black hole binaries.}
\newline{}
N. Steinle, H. Middleton, C. J. Moore, S. Chen, A. Klein, G. Pratten, \textbf{R. Buscicchio}, E. Finch, A. Vecchio.
\newline{}
\href{https://academic.oup.com/mnras/article/525/2/2851/7244712}{\mnras 525 2 (2023)}. \href{https://arxiv.org/abs/2305.05955}{arXiv:2305.05955 [astro-ph.HE].}
\vspace{0.09cm}\\
%
\textbf{16.} & & \textit{Parameter estimation of binary black holes in the endpoint of the up-down instability.}
\newline{}
V. De Renzis, D. Gerosa, M. Mould, \textbf{R. Buscicchio}, L. Zanga.
\newline{}
\href{https://journals.aps.org/prd/abstract/10.1103/PhysRevD.108.024024}{\prd 108 (2023) 024024}. \href{https://arxiv.org/abs/2304.13063}{arXiv:2304.13063 [gr-qc].}
\vspace{0.09cm}\\
%
\textbf{15.} & & \textit{Improved detection statistics for non Gaussian gravitational wave stochastic backgrounds.}
\newline{}
M. Ballelli, \textbf{R. Buscicchio}, B. Patricelli, A. Ain, G. Cella.
\newline{}
\href{https://journals.aps.org/prd/abstract/10.1103/PhysRevD.107.124044}{\prd 107 (2023) 124044}. \href{https://arxiv.org/abs/2212.10038}{arXiv:2212.10038 [gr-qc].}
\vspace{0.09cm}\\
%
\textbf{14.} & & \textit{Detecting non-Gaussian gravitational wave backgrounds: a unified framework.}
\newline{}
\textbf{R. Buscicchio}, A. Ain, M. Ballelli, G. Cella, B. Patricelli.
\newline{}
\href{https://journals.aps.org/prd/abstract/10.1103/PhysRevD.107.063027}{\prd 107 (2023) 063027}. \href{https://arxiv.org/abs/2209.01400}{arXiv:2209.01400 [gr-qc].}
\vspace{0.09cm}\\
%
\textbf{13.} & & \textit{Detectability of a spatial correlation between stellar-mass black hole mergers and Active Galactic Nuclei in the Local Universe.}
\newline{}
N. Veronesi, E.M. Rossi, S. van Velzen, \textbf{R. Buscicchio}.
\newline{}
\href{https://academic.oup.com/mnras/article/514/2/2092/6587069}{\mnras 514 2 (2023)}. \href{https://arxiv.org/abs/2203.05907}{arXiv:2203.05907 [astro-ph.HE].}
\vspace{0.09cm}\\
%
\textbf{12.} & & \textit{Bayesian parameter estimation of stellar-mass black-hole binaries with LISA.}
\newline{}
\textbf{R. Buscicchio}, A. Klein, E. Roebber, C. J. Moore, D. Gerosa, E. Finch, A. Vecchio.
\newline{}
\href{https://journals.aps.org/prd/abstract/10.1103/PhysRevD.104.044065}{\prd 104 (2021) 044065}. \href{https://arxiv.org/abs/2106.05259}{arXiv:2106.05259 [astro-ph.HE].}
\vspace{0.09cm}\\
%
\textbf{11.} & & \textit{An Interactive Gravitational-Wave Detector Model for Museums and Fairs.}
\newline{}
S. J. Cooper, A. C. Green, H. R. Middleton, C. P. L. Berry, \textbf{R. Buscicchio}, E. Butler, C. J. Collins, C. Gettings, D. Hoyland, A. W. Jones, J. H. Lindon, I. Romero-Shaw, S. P. Stevenson, E. P. Takeva, S. Vinciguerra, A. Vecchio, C. M. Mow-Lowry, A. Freise.
\newline{}
\href{https://pubs.aip.org/aapt/ajp/article/89/7/702/1056907/An-interactive-gravitational-wave-detector-model}{\ajp 89 (2021) 702–712}. \href{https://arxiv.org/abs/2004.03052}{arXiv:2004.03052 [physics.ed-ph].}
\vspace{0.09cm}\\
%
\textbf{10.} & & \textit{Evidence for hierarchical black hole mergers in the second LIGO--Virgo gravitational-wave catalog.}
\newline{}
C. Kimball, C. Talbot, C.P.L. Berry, M. Zevin, E. Thrane, V. Kalogera, \textbf{R. Buscicchio}, M. Carney, T. Dent, H. Middleton, E. Payne, J. Veitch, D. Williams .
\newline{}
\href{https://iopscience.iop.org/article/10.3847/2041-8213/ac0aef}{\apjl 915 (2021) L35}. \href{https://arxiv.org/abs/2011.05332}{arXiv:2011.05332 [astro-ph.HE].}
\vspace{0.09cm}\\
%
\textbf{9.} & & \textit{Testing general relativity with gravitational-wave catalogs: the insidious nature of waveform systematics.}
\newline{}
C. J. Moore, E. Finch, \textbf{R. Buscicchio}, D. Gerosa.
\newline{}
\href{https://www.sciencedirect.com/science/article/pii/S2589004221005459}{iScience 24 (2021) 102577}. \href{https://arxiv.org/abs/2103.16486}{arXiv:2103.16486   [gr-qc].}
\vspace{0.09cm}\\
%
\textbf{8.} & & \textit{LoCuSS: The splashback radius of massive galaxy clusters and its dependence on cluster merger history.}
\newline{}
M. Bianconi, \textbf{R. Buscicchio}, G. P. Smith, S. L. McGee, C.P. Haines, A. Finoguenov, A. Babul.
\newline{}
\href{https://iopscience.iop.org/article/10.3847/1538-4357/abebd7}{\apj 911 (2021) 136}. \href{https://arxiv.org/abs/2010.05920}{arXiv:2010.05920 [astro-ph.GA].}
\vspace{0.09cm}\\
%
\textbf{7.} & & \textit{Search for Black Hole Merger Families.}
\newline{}
D. Veske, A. G. Sullivan, Z. Marka, I. Bartos, K. R. Corley, J. Samsing, \textbf{R. Buscicchio}, S. Marka.
\newline{}
\href{https://iopscience.iop.org/article/10.3847/2041-8213/abd721}{\apjl 907 (2021) L48}. \href{https://arxiv.org/abs/2011.06591}{arXiv:2011.06591 [astro-ph.HE].}
\vspace{0.09cm}\\
%
\textbf{6.} & & \textit{Constraining the lensing of binary black holes from their stochastic background.}
\newline{}
\textbf{R. Buscicchio}, C. J. Moore, G. Pratten, P. Schmidt, M. Bianconi, A. Vecchio.
\newline{}
\href{https://journals.aps.org/prl/abstract/10.1103/PhysRevLett.125.141102}{\prl 125 (2020) 141102}. \href{https://arxiv.org/abs/2006.04516}{arXiv:2006.04516 [astro-ph.CO].}
\vspace{0.09cm}\\
%
\textbf{5.} & & \textit{Constraining the lensing of binary neutron stars from their stochastic background.}
\newline{}
\textbf{R. Buscicchio}, C. J. Moore, G. Pratten, P. Schmidt, A. Vecchio.
\newline{}
\href{https://journals.aps.org/prd/abstract/10.1103/PhysRevD.102.081501}{\prd 102 (2020) 081501 }. \href{https://arxiv.org/abs/2008.12621}{arXiv:2008.12621 [astro-ph.HE].}
\vspace{0.09cm}\\
%
\textbf{4.} & & \textit{Measuring precession in asymmetric compact binaries.}
\newline{}
G. Pratten, P. Schmidt, \textbf{R. Buscicchio}, L. M. Thomas.
\newline{}
\href{https://journals.aps.org/prresearch/abstract/10.1103/PhysRevResearch.2.043096}{\prr 2 (2020) 043096}. \href{https://arxiv.org/abs/2006.16153}{arXiv:2006.16153 [gr-qc].}
\vspace{0.09cm}\\
%
\textbf{3.} & & \textit{Populations of double white dwarfs in Milky Way satellites and their detectability with LISA.}
\newline{}
V. Korol, S. Toonen, A. Klein, V. Belokurov, F. Vincenzo, \textbf{R. Buscicchio}, D. Gerosa, C. J. Moore, E. Roebber, E. M. Rossi, A. Vecchio.
\newline{}
\href{https://www.aanda.org/articles/aa/abs/2020/06/aa37764-20/aa37764-20.html}{\aap 638 (2020) A153}. \href{https://arxiv.org/abs/2002.10462}{arXiv:2002.10462 [astro-ph.GA].}
\vspace{0.09cm}\\
%
\textbf{2.} & & \textit{Milky Way satellites shining bright in gravitational waves.}
\newline{}
E. Roebber, \textbf{R. Buscicchio}, A. Vecchio, C. J. Moore, A. Klein, V. Korol, S. Toonen, D. Gerosa, J. Goldstein, S. M. Gaebel, T. E. Woods.
\newline{}
\href{https://iopscience.iop.org/article/10.3847/2041-8213/ab8ac9}{\apjl 894 (2020) L15}. \href{https://arxiv.org/abs/2002.10465}{arXiv:2002.10465 [astro-ph.GA].}
\vspace{0.09cm}\\
%
\textbf{1.} & & \textit{Label Switching Problem in Bayesian Analysis for Gravitational Wave Astronomy.}
\newline{}
\textbf{R. Buscicchio}, E. Roebber, J. M. Goldstein, C. J. Moore .
\newline{}
\href{https://journals.aps.org/prd/abstract/10.1103/PhysRevD.100.084041}{\prd 100 (2019) 084041}. \href{https://arxiv.org/abs/1907.11631}{arXiv:1907.11631 [astro-ph.IM].}
\vspace{0.09cm}\\
%
\end{longtable} }
\textcolor{color1}{\textbf{Collaboration papers in major peer-reviewed journals, which I have substantially contributed to.:}}
\vspace{-0.5cm}

\cvitem{}{\small\hspace{-1cm}\begin{longtable}{rp{0.3cm}p{15.8cm}}
%
\textbf{13.} & & \textit{Search for gravitational-lensing signatures in the full third observing run of the LIGO-Virgo network.}
\newline{}
LIGO Scientific Collaboration, Virgo Collaboration, KAGRA collaboration.
\newline{}
\href{https://iopscience.iop.org/article/10.3847/1538-4357/ad3e83}{\apj 970 (2021) 191}. \href{https://arxiv.org/abs/2304.08393}{arXiv:2304.08393 [gr-qc].}
\vspace{0.09cm}\\
%
\textbf{12.} & & \textit{GWTC-2.1: Deep Extended Catalog of Compact Binary Coalescences Observed by LIGO and Virgo During the First Half of the Third Observing Run.}
\newline{}
LIGO Scientific Collaboration, Virgo Collaboration, KAGRA collaboration.
\newline{}
\href{https://journals.aps.org/prd/abstract/10.1103/PhysRevD.109.022001}{\prd 109 (2024) 022001}. \href{https://arxiv.org/abs/2108.01045}{arXiv:2108.01045 [gr-qc].}
\vspace{0.09cm}\\
%
\textbf{11.} & & \textit{The population of merging compact binaries inferred using gravitational waves through GWTC-3.}
\newline{}
LIGO Scientific Collaboration, Virgo Collaboration, KAGRA collaboration.
\newline{}
\href{https://journals.aps.org/prx/abstract/10.1103/PhysRevX.13.011048}{\prx 13 (2021) 011048}. \href{https://arxiv.org/abs/2111.03634}{arXiv:2111.03634 [astro-ph.HE].}
\vspace{0.09cm}\\
%
\textbf{10.} & & \textit{Tests of General Relativity with GWTC-3.}
\newline{}
LIGO Scientific Collaboration, Virgo Collaboration, KAGRA collaboration.
\newline{}
\href{https://journals.aps.org/prd/accepted/17075Qf4Z7b11729787e85f1c18faca230d51e013}{\prd (in press)}. \href{https://arxiv.org/abs/2112.06861}{arXiv:2112.06861 [gr-qc].}
\vspace{0.09cm}\\
%
\textbf{9.} & & \textit{Search for lensing signatures in the gravitational-wave observations from the first half of LIGO-Virgo's third observing run.}
\newline{}
LIGO Scientific Collaboration, Virgo Collaboration, KAGRA collaboration.
\newline{}
\href{https://iopscience.iop.org/article/10.3847/1538-4357/ac23db}{\apjl (2021) 923}. \href{https://arxiv.org/abs/2105.06384}{arXiv:2105.06384 [gr-qc].}
\vspace{0.09cm}\\
%
\textbf{8.} & & \textit{GWTC-3: Compact Binary Coalescences Observed by LIGO and Virgo During the Second Part of the Third Observing Run.}
\newline{}
LIGO Scientific Collaboration, Virgo Collaboration, KAGRA collaboration.
\newline{}
\href{https://journals.aps.org/prx/abstract/10.1103/PhysRevX.13.041039}{\prx 13 (2023) 041039}. \href{https://arxiv.org/abs/2111.03606}{arXiv:2111.03606 [gr-qc].}
\vspace{0.09cm}\\
%
\textbf{7.} & & \textit{Observation of gravitational waves from two neutron star-black hole coalescences.}
\newline{}
LIGO Scientific Collaboration, Virgo Collaboration.
\newline{}
\href{https://iopscience.iop.org/article/10.3847/2041-8213/ac082e}{\apjl, 915, L5 (2021)}. \href{https://arxiv.org/abs/2106.15163}{arXiv:2106.15163 [astro-ph.HE].}
\vspace{0.09cm}\\
%
\textbf{6.} & & \textit{GWTC-2: Compact Binary Coalescences Observed by LIGO and Virgo During the First Half of the Third Observing Run.}
\newline{}
LIGO Scientific Collaboration, Virgo Collaboration.
\newline{}
\href{https://journals.aps.org/prx/abstract/10.1103/PhysRevX.11.021053}{\prx 11 (2021) 021053}. \href{https://arxiv.org/abs/2010.14527}{arXiv:2010.14527 [gr-qc].}
\vspace{0.09cm}\\
%
\textbf{5.} & & \textit{Population Properties of Compact Objects from the Second LIGO-Virgo Gravitational-Wave Transient Catalog.}
\newline{}
LIGO Scientific Collaboration, Virgo Collaboration.
\newline{}
\href{https://iopscience.iop.org/article/10.3847/2041-8213/abe949}{\apjl 913 (2021) L7}. \href{https://arxiv.org/abs/2010.14533}{arXiv:2010.14533 [astro-ph.HE].}
\vspace{0.09cm}\\
%
\textbf{4.} & & \textit{Upper Limits on the Isotropic Gravitational-Wave Background from Advanced LIGO's and Advanced Virgo's Third Observing Run.}
\newline{}
LIGO Scientific Collaboration, Virgo Collaboration, KAGRA collaboration.
\newline{}
\href{https://journals.aps.org/prd/abstract/10.1103/PhysRevD.104.022004}{\prd 104 (2021) 022004}. \href{https://arxiv.org/abs/2101.12130}{arXiv:2101.12130 [gr-qc].}
\vspace{0.09cm}\\
%
\textbf{3.} & & \textit{Binary Black Hole Population Properties Inferred from the First and Second Observing Runs of Advanced LIGO and Advanced Virgo .}
\newline{}
LIGO Scientific Collaboration, Virgo Collaboration.
\newline{}
\href{https://iopscience.iop.org/article/10.3847/2041-8213/ab3800}{\apj 882 (2019)  L24}. \href{https://arxiv.org/abs/1811.12940}{arXiv:1811.12940 [astro-ph.HE].}
\vspace{0.09cm}\\
%
\textbf{2.} & & \textit{Properties and astrophysical implications of the 150 Msun binary black hole merger GW190521.}
\newline{}
LIGO Scientific Collaboration, Virgo Collaboration.
\newline{}
\href{https://iopscience.iop.org/article/10.3847/2041-8213/aba493}{\apjl 900 (2020) L13}. \href{https://arxiv.org/abs/2009.01190}{arXiv:2009.01190 [astro-ph.HE].}
\vspace{0.09cm}\\
%
\textbf{1.} & & \textit{GW190521: A Binary Black Hole Merger with a Total Mass of 150 $M_\odot$.}
\newline{}
LIGO Scientific Collaboration, Virgo Collaboration.
\newline{}
\href{https://journals.aps.org/prl/abstract/10.1103/PhysRevLett.125.101102}{\prl 125 (2020) 101102}. \href{https://arxiv.org/abs/2009.01075}{arXiv:2009.01075 [gr-qc].}
\vspace{0.09cm}\\
%
\end{longtable} }
\textcolor{color1}{\textbf{PhD thesis, technical reports.:}}
\vspace{-0.5cm}

\cvitem{}{\small\hspace{-1cm}\begin{longtable}{rp{0.3cm}p{15.8cm}}
%
\textbf{2.} & & \textit{LISA - Laser Interferometer Space Antenna - Definition Study Report.}
\newline{}
The European Space Agency.
\newline{}
\href{https://www.cosmos.esa.int/documents/15452792/15452811/LISA_DEFINITION_STUDY_REPORT_ESA-SCI-DIR-RP-002_Public+(1).pdf/2deb7646-dccd-ae0d-75c1-b2e16df584cf?t=1707166191449}{ESA-SCI-DIR-RP-002}. 
\vspace{0.09cm}\\
%
\textbf{1.} & & \textit{Topics in Bayesian population inference for gravitational wave astronomy.}
\newline{}
\textbf{R. Buscicchio}.
\newline{}
\href{https://etheses.bham.ac.uk//id/eprint/12288/}{PhD thesis}. 
\vspace{0.09cm}\\
%
\end{longtable} }

\section{Full presentation list}\vspace{0.2cm} 

Invited talks marked with *.
\vspace{0.2cm}

\textcolor{color1}{\textbf{Talks at conferences:}}
\vspace{-0.5cm}

\cvitem{}{\small\hspace{-1cm}\begin{longtable}{rp{0.3cm}p{15.8cm}}
%
\textbf{29.} & * & \textit{Emergence of Milky Way structure in the first year of LISA data.}
\newline{}
CERN UniGe Gravitational Wave meeting, Geneva, Switzerland, 2025/05/23.
\vspace{0.05cm}\\
%
\textbf{28.} &  & \textit{LISA stellar-mass black holes informed by the GWTC-3 population: event rates and parameters reconstruction.}
\newline{}
LISA Astrophysics Working Group Meeting 2024, Garching, Germany, 2024/11/05.
\vspace{0.05cm}\\
%
\textbf{27.} & * & \textit{Astrophysics panel session.}
\newline{}
GRASP: Gravity Shape Pisa 2024, Pisa, Italy, 2024/10/24.
\vspace{0.05cm}\\
%
\textbf{26.} & * & \textit{Beyond Gauss? A more accurate model for LISA astrophysical noise sources.}
\newline{}
Kavli Institute for Cosmology Seminars, Cambridge, United Kingdom, 2024/10/14.
\vspace{0.05cm}\\
%
\textbf{25.} & * & \textit{Beyond Gauss? A more accurate model for LISA astrophysical noise sources.}
\newline{}
Heterogeneous Data and Large Representation Models in Science, Toulouse, France, 2024/10/01.
\vspace{0.05cm}\\
%
\textbf{24.} &  & \textit{LISA stellar-mass black holes informed by the GWTC-3 population: event rates and parameters reconstruction.}
\newline{}
15th International LISA Symposium, Dublin, Ireland, 2024/07/08.
\vspace{0.05cm}\\
%
\textbf{23.} & * & \textit{LISA data analysis: from the stochastic background to the Milky Way.}
\newline{}
11th LISA Cosmology Working Group Workshop, Porto, Portugal, 2024/06/19.
\vspace{0.05cm}\\
%
\textbf{22.} & * & \textit{An introduction to Bayesian Inference.}
\newline{}
International Pulsar Timing Array Student Week, Milan, Italy, 2024/06/17.
\vspace{0.05cm}\\
%
\textbf{21.} & * & \textit{Statistical challenges in LISA data analysis.}
\newline{}
LAUTARO joint meeting, GSSI-University of Milano-Bicocca, Milano, Italy, 2024/04/17.
\vspace{0.05cm}\\
%
\textbf{20.} &  & \textit{From mHz to kHz: stochastic background implications on astrophysical sources and population reconstruction.}
\newline{}
LISA Astrophysics working group workshop, University of Milano-Bicocca, Milano, Italy, 2023/09/13.
\vspace{0.05cm}\\
%
\textbf{19.} &  & \textit{Non-gaussian gravitational wave backgrounds across the GW spectrum.}
\newline{}
XXV Sigrav conference on general relativity and gravitation, SISSA, Trieste, Italy, 2023/09/04.
\vspace{0.05cm}\\
%
\textbf{18.} & * & \textit{LISA SGWB data analysis (session chair).}
\newline{}
Data Analysis Challenges for SGWB Workshop, CERN, Geneva, Switzerland, 2023/07/19.
\vspace{0.05cm}\\
%
\textbf{17.} & * & \textit{Global Fit and foregrounds.}
\newline{}
LISA SGWB detection brainstorming, Univ. of Geneva, Geneva, Switzerland, 2023/07/17.
\vspace{0.05cm}\\
%
\textbf{16.} & * & \textit{Beyond functional forms: non-parametric methods. (panelist talk).}
\newline{}
Gravitational-wave populations: What's next?, University of Milano-Bicocca, Milan, Italy, 2023/07/01.
\vspace{0.05cm}\\
%
\textbf{15.} &  & \textit{The last three years : multiband gravitational-wave observations of stellar-mass binary black holes.}
\newline{}
LISA Astrophysics working group workshop, University of Birmingham, Birmingham, UK, 2022/06/23.
\vspace{0.05cm}\\
%
\textbf{14.} &  & \textit{The last three years : multiband gravitational-wave observations of stellar-mass binary black holes.}
\newline{}
American Physical Society (APS) April meeting, New York (NY), USA, 2022/04/12.
\vspace{0.05cm}\\
%
\textbf{13.} &  & \textit{Bayesian parameter estimation of stellar-mass black-hole binaries with LISA.}
\newline{}
XXIV Sigrav conference on general relativity and gravitation, Urbino, Italy, 2021/09/08.
\vspace{0.05cm}\\
%
\textbf{12.} &  & \textit{Chirp: a web and smartphone application for visualization of gravitational-wave alerts.}
\newline{}
14th Amaldi Conference on Gravitational Waves, (online), 2021/07/21.
\vspace{0.05cm}\\
%
\textbf{11.} & * & \textit{Search for lensing signatures in the gravitational-wave observations from the first half of LIGO-Virgo’s third observing run.}
\newline{}
2nd EPS conference on gravitation, (online, on behalf of LVK), 2021/05/27.
\vspace{0.05cm}\\
%
\textbf{10.} & * & \textit{Bayesian parameter estimation of stellar-mass black-hole binaries with LISA.}
\newline{}
LISA Data Challenge meeting, (online), 2021/06/17.
\vspace{0.05cm}\\
%
\textbf{9.} & * & \textit{Search for lensing signatures in the gravitational-wave observations from the first half of LIGO-Virgo’s third observing run.}
\newline{}
Webinar on behalf of the LVK collaboration, (online), 2021/05/27.
\vspace{0.05cm}\\
%
\textbf{8.} &  & \textit{Milky Way Satellites Shining Bright in Gravitational Waves.}
\newline{}
13th LISA Symposium, (online), 2020/09/13.
\vspace{0.05cm}\\
%
\textbf{7.} &  & \textit{Constraining the Lensing of Binary Black Holes from Their Stochastic Background.}
\newline{}
LISA Sprint workshop, CCA, Flatiron Institute, New York (NY), USA, 2020/03/04.
\vspace{0.05cm}\\
%
\textbf{6.} &  & \textit{Multiple source detection in GW astronomy: the label switching problem.}
\newline{}
30th Texas Symposium, University of Portsmouth, Portsmouth, UK, 2019/12/12.
\vspace{0.05cm}\\
%
\textbf{5.} &  & \textit{Non-gaussian Stochastic background search with importance sampling.}
\newline{}
LIGO, Virgo, KAGRA September meeting, Warsaw, Poland, 2019/09/01.
\vspace{0.05cm}\\
%
\textbf{4.} &  & \textit{An improved detector for non-Gaussian stochastic background.}
\newline{}
Stochastic Background Data Analysis for LISA meeting, Instituto de Fisica Teorica, Madrid, Spain, 2019/06/01.
\vspace{0.05cm}\\
%
\textbf{3.} &  & \textit{Hierarchical nonparametric density estimation for population inference.}
\newline{}
LIGO, Virgo, KAGRA March meeting, Winsconsin, USA, 2019/03/18.
\vspace{0.05cm}\\
%
\textbf{2.} &  & \textit{Fast Evaluation of Campbell processes N–point correlation functions.}
\newline{}
Astro Hack Week: Data Science for Next-Generation Astronomy, Lorentz Center, Leiden, The Netherlands, 2018/08/01.
\vspace{0.05cm}\\
%
\textbf{1.} &  & \textit{Stochastic Gravitational Wave Background Data Analysis for Radler.}
\newline{}
5th LISA Cosmology Working Group workshop, Physicum, University of Helsinki, Helsinki, Finland, 2018/06/01.
\vspace{0.05cm}\\
%
\end{longtable} }
\textcolor{color1}{\textbf{Talks at department seminars:}}
\vspace{-0.5cm}

\cvitem{}{\small\hspace{-1cm}\begin{longtable}{rp{0.3cm}p{15.8cm}}
%
\textbf{10.} & * & \textit{Fast LISA inference using Gaussian processes.}
\newline{}
University of Geneva, Geneva, Switzerland, 2025/05/21.
\vspace{0.05cm}\\
%
\textbf{9.} & * & \textit{Emergence of Milky Way structure in the first year of LISA data.}
\newline{}
Department of Physics, University of Pisa, Pisa, Italy, 2025/05/16.
\vspace{0.05cm}\\
%
\textbf{8.} & * & \textit{Statistical challenges in GW inference: an application of field theory to direct population reconstruction in LISA.}
\newline{}
APP seminar, SISSA, Trieste, Italy, 2024/05/06.
\vspace{0.05cm}\\
%
\textbf{7.} & * & \textit{GRAF: Gravitational waves data and global fit.}
\newline{}
Department of Physics, University of Milano-Bicocca, Milan, Italy, 2023/12/14.
\vspace{0.05cm}\\
%
\textbf{6.} & * & \textit{LISA global inference: statistical and modelling challenges  for the Milky Way.}
\newline{}
Max Planck Institute for Astrophysics, Garching, Germany, 2023/11/29.
\vspace{0.05cm}\\
%
\textbf{5.} & * & \textit{LISA Global inference: modelling, statistical, and computational challenges.}
\newline{}
Department of Physics, University of Pisa, Pisa, Italy, 2023/10/04.
\vspace{0.05cm}\\
%
\textbf{4.} & * & \textit{Gravitational waves in the many sources, many detectors era.}
\newline{}
Institute for Mathematics and Physics, University of Stavanger, Stavanger, Norway, 2022/09/29.
\vspace{0.05cm}\\
%
\textbf{3.} & * & \textit{Stellar mass binary black holes : what, when, and where.}
\newline{}
Astroparticule et cosmologie, Universitè Paris Citè, Paris, France, 2022/06/12, (online).
\vspace{0.05cm}\\
%
\textbf{2.} & * & \textit{The last three years: multiband gravitational-wave observations of stellar-mass binary black holes.}
\newline{}
Physics Department, Columbia University, New York (NY), USA, 2022/04/07.
\vspace{0.05cm}\\
%
\textbf{1.} & * & \textit{Set the alarm : Bayesian parameter estimation of stellar-mass black-hole binaries with LISA.}
\newline{}
Sun Yat-sen University, Zhuhai, China, 2021/07/30, (online).
\vspace{0.05cm}\\
%
\end{longtable} }
\textcolor{color1}{\textbf{Outreach \& public engagement talks:}}
\vspace{-0.5cm}

\cvitem{}{\small\hspace{-1cm}\begin{longtable}{rp{0.3cm}p{15.8cm}}
%
\textbf{5.} &  & \textit{Onde gravitazionali: ascoltare l'Universo anzich'e solo guardarlo.}
\newline{}
University of Milano-Bicocca, Milan, Italy, 2024.
\vspace{0.05cm}\\
%
\textbf{4.} &  & \textit{An orchestra of lasers and gravitational waves.}
\newline{}
Pint of Science 2024, Milan, Italy, 2024.
\vspace{0.05cm}\\
%
\textbf{3.} &  & \textit{Gravitational-waves in space and on Earth.}
\newline{}
Manchester Museum of Science and Industry, Manchester, UK, 2018.
\vspace{0.05cm}\\
%
\textbf{2.} &  & \textit{An orchestra of lasers and gravitational waves.}
\newline{}
Manchester Museum of Science and Industry, Manchester, UK, 2018.
\vspace{0.05cm}\\
%
\textbf{1.} &  & \textit{A Universe of waves.}
\newline{}
Science Caf'e, Italy, 2018.
\vspace{0.05cm}\\
%
\end{longtable} }
%mark_CVshort

\end{document}
