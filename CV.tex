\documentclass[colorlinks,linkcolor=teal,a4paper,11pt]{moderncv}
	
\usepackage{graphicx}
\usepackage{amssymb}
\usepackage{amsmath}
\usepackage[utf8]{inputenc}
\usepackage{longtable}
\usepackage{xcolor}
\usepackage{xspace}
\newcommand{\rsquo}{{\tt\char'023}}

\moderncvstyle{banking}
\moderncvcolor{green}

\definecolor{color1}{rgb}{0.0, 0.6, 0.6}
\definecolor{mark_color}{rgb}{0.5, 0.5, 0.5}

%\usepackage{cmbright}

\usepackage[sfdefault,lf]{carlito}
\usepackage[T1]{fontenc}
\renewcommand*\oldstylenums[1]{\carlitoOsF #1}

\usepackage[top=1.5cm,bottom=2cm,left=2cm,right=2cm,bindingoffset=0cm]{geometry}
\setlength{\hintscolumnwidth}{3cm}
\usepackage{enumitem}
\setlist{nolistsep}

\makeatletter
\renewcommand*{\bibliographyitemlabel}{\@biblabel{\arabic{enumiv}}}
\makeatother

\newcommand{\mnras}{Monthly Notices of the Royal Astronomical Society\xspace}
\newcommand{\mnrasl}{Monthly Notices of the Royal Astronomical Society Letters\xspace}
\newcommand{\jcap}{Journal of Cosmology and Astroparticle Physics\xspace}
\newcommand{\prd}{Physical Review D\xspace}
\newcommand{\prdl}{Physical Review D Letters\xspace}
\newcommand{\prdrc}{Physical Review D Rapid Communications\xspace}
\newcommand{\prx}{Physical Review X\xspace}
\newcommand{\prl}{\textbf{Physical Review Letters}\xspace}
\newcommand{\prlplain}{{Physical Review Letters\xspace}}
\newcommand{\cqg}{Classical and Quantum Gravity\xspace}
\newcommand{\aap}{Astronomy \& Astrophysics\xspace}
\newcommand{\prr}{Physical Review Research\xspace}
\newcommand{\apj}{Astrophysical Journal\xspace}
\newcommand{\apjl}{Astrophysical Journal Letters\xspace}
\newcommand{\ajp}{American Journal of Physics\xspace}
\newcommand{\grg}{General Relativity and Gravitation\xspace}
\newcommand{\natastro}{Nature Astronomy\xspace}
\newcommand{\lrr}{Living Reviews in Relativity\xspace}


\long\def\suppress#1\endsuppress{%
  \begingroup%
    \tracinglostchars=0%
    \let\selectfont=\nullfont
    \nullfont #1\endgroup}

\fancypagestyle{headonly}{
\fancyfoot{}
\fancyfoot[r]{\textcolor{color1}{\thepage}}
\fancyhead{}
}

\newcommand{\mytitle}[1]{\title{#1\vspace{0.15cm}}}

\firstname{Riccardo}
\familyname{Buscicchio}

\extrainfo{\normalsize riccardo.buscicchio@unimib.it $\;\;\bullet\;\;$ \href{http://www.riccardobuscicchio.com}{www.riccardobuscicchio.com}  $\;\;\bullet\;\;$ \today}


\mytitle{Curriculum Vit\ae}

\begin{document}
\hypersetup{urlcolor=color1}
		
\pagestyle{headonly}

\makecvtitle

%mark_CVshort
\cvitem{}{\emph{\vspace{-1.0cm}\\
$\quad$ Astrofisica relativistica, sviluppo di algoritmi di analisi dati per inferenza Bayesiana e frequentista. Principali applicazioni: modellizzazione di missioni spaziali, rivelazione di segnali e stima dei parametri per astronomia ad onde gravitazionali, inferenza gerarchica di popolazione, ricerca di fondi stocastici. Modellizzazione della morfologia della Via Lattea, caratterizzazione di precursori di supernovae, inferenza sui modelli di accresscimento di buchi neri supermassicci.  
}}
%mark_CVshort

%mark_CVshort
\section{Contatti}

%mark_CVshort
\cvitem{Email}{\href{mailto:riccardo.buscicchio@unimib.it}{riccardo.buscicchio@unimib.it}}
%mark_CVshort
\cvitem{Indirizzo}{Universit\`{a} degli Studi di Milano-Bicocca, Piazza della Scienza 3, 20126 Milano, Italia.}
\cvitem{Nazionalità}{Italia}
\cvitem{Sito web \& e pubblicazioni}{
	\href{https://www.riccardobuscicchio.com/}{www.riccardobuscicchio.com} -- \href{https://arxiv.org/a/buscicchio_r_1.html}{\textsc{arXiv}} --
	\href{https://orcid.org/0000-0002-7387-6754}{\textsc{ORCID}}
	}
%mark_CVshort

%\cvitem{Citizenship}{Italy, EU.}

\section{Incarichi accademici}

\cventry{2021 - 2024}{Assegnista di ricerca, Dipartimento di Fisica ``G.Occhialini''}{Universit\`{a} degli Studi di Milano-Bicocca}{Milano, Italia}{}{}
\vspace{-0.1cm}
\begin{tabular}{rcl}
&\hspace{0.4cm} &$\circ\;\;${\textit{Attività di ricerca}}: sviluppo del segmento a terra per la analisi dati di LISA per\\
&\hspace{0.4cm} &\phantom{$\circ\;\;${\textit{Attività di ricerca}} } l'Agenzia Spaziale Italiana (Fase A).
\end{tabular}

\cventry{2024 - current}{Assegnista di ricerca, Dipartimento di Fisica ``G.Occhialini''}{Universit\`{a} degli Studi di Milano-Bicocca}{Milano, Italia}{}{}
\vspace{-0.1cm}
\begin{tabular}{rcl}
&\hspace{0.4cm} &$\circ\;\;${\textit{Attività di ricerca}}: sviluppo del segmento a terra per la analisi dati di LISA per\\
& & \phantom{$\circ\;\;${\textit{Attività di ricerca}} } l'Agenzia Spaziale Italiana (Fase A).
\end{tabular}

\vspace{-0.2cm}
\section{Istruzione}

\cventry{2017-13/07/2022}{Ph.D., School of Physics \& Astronomy}{University of Birmingham}{Birmingham, Regno Unito}{}{}
\vspace{-0.1cm}
\begin{tabular}{rcl}
&\hspace{0.4cm} &$\circ\;\;${\textit{Supervisore}}: A.~Vecchio. La tesi ha prodotto 6 pubblicazioni a lista di autori breve.
\\
&\hspace{0.4cm} &$\circ\;\;${\textit{Titolo della tesi}}:
Topics in Bayesian population inference for Gravitational Wave Astronomy
\end{tabular}
\vspace{0.2cm}

%mark_CVshort
Questa tesi esplora diversi argomenti relativi all'inferenza bayesiana nell'astronomia a onde gravitazionali. Dall'inferenza gerarchica sulle popolazioni di buchi neri binari di massa stellare, allo sviluppo di routines di fit globale di onde gravitazionali per la stima dei parametri da sorgenti osservate mediante interferometri spaziali. Vengono inoltre affrontati i seguenti temi: popolazione di nane bianche binarie nelle galassie satellite della Via Lattea; vincoli dai fondi stocastici sul lensing di onde gravitazionali provenienti dalla fusione di binarie di stelle di neutroni e buchi neri; tecniche statistiche per l'inferenza congiunta su sorgenti multiple indistinguibili.
%mark_CVshort

\vspace{0.2cm}
\cventry{2013-2016}{Laurea Magistrale in Fisica Teorica}{\newline Universit\`{a} degli Studi di Pisa}{Pisa, Italia}{}{}
\vspace{-0.1cm}
\begin{tabular}{rcl}
%mark_CVshort
&\hspace{0.4cm} &$\circ\;\;${\textit{Voto conseguito}}: 110/110\\
%mark_CVshort
&\hspace{0.4cm} &$\circ\;\;${\textit{Supervisore}}: G.~Cella. La tesi ha prodotto una pubblicazione a lista di autori breve.\\
&\hspace{0.4cm} &$\circ\;\;${\textit{Titolo della tesi}}: 
An improved detector for non-gaussian stochastic background of gravitational waves.
\end{tabular}
\vspace{0.2cm}

%mark_CVshort
In questa tesi si è sviluppato il formalismo funzionale, inspirato a quello dei processi stocastici e della teoria dei campi classica, per realizzare un algoritmo migliorato per la rivelazione di fondi stocastici non gaussiani di onde gravitazionali.
%mark_CVshort

\vspace{0.2cm}
\cventry{Giu-Sett 2013}{Programma di internship INFN-NSF}{Columbia University}{New York NY, Stati Uniti}{}{}
\vspace{-0.1cm}
\begin{tabular}{rcl}
&\hspace{0.4cm} &$\circ\;\;${\textit{Supervisori}}: S.~Marka, I.~Bartos.
\end{tabular}
\vspace{0.2cm}

%mark_CVshort
Abbiamo stimato il contributo al livello di rumore per rivelatori di onde gravitazionali di seconda e terza generazione dovuto a sciami di raggi cosmici primari e secondari che attraversano su specchi per inteferometria da osservatori gravitazionali terrestri.
%mark_CVshort

%mark_CVshort
\vspace{0.2cm}
\cventry{2008-2012}{Laurea Triennale in Fisica}{Universit\`{a} degli Studi di Pisa}{Pisa, Italia}{}{}
\vspace{-0.1cm}
\begin{tabular}{rcl}
&\hspace{0.4cm} &$\circ\;\;${\textit{Voto conseguito}}: 109/110.\\
&\hspace{0.4cm} &$\circ\;\;${\textit{Titolo della tesi}}: Banchi di template per la rivelazione delle onde gravitazionali:\\
&\hspace{0.4cm} &\phantom{$\circ\;\;${\textit{Titolo della tesi}} } un'applicazione dell'Information Geometry.
\end{tabular}
\vspace{0.2cm}
%mark_CVshort


%mark_CVshort
Questa tesi ha esplorato l'idea di utilizzare il formalismo della geometria differenziale (come definito nel contesto della teoria dell'informazione) per sviluppare un algoritmo di posizionamento dei modelli di segnale (template) nello spazio dei parametri delle sorgenti, sfruttando la struttura di varietà indotta dalla statistica di rivelazione.
%mark_CVshort

\section{Indicatori bibliometrici}

\cvitem{}{\begin{tabular}{rcl}
\textcolor{mark_color}{\textbf{Pubblicazioni}}: & \hspace{0.3cm} & \\
&\textbf{27\, } & pubblicazioni short-author in riviste internazionali peer-reviewed\\
& & (di cui \textbf{7}\, articoli a primo autore e \textbf{5}\, di studenti supervisionati).\\
&\textbf{13} & articoli di collaborazione con contributo significativo in riviste internazionali peer-reviewed\\
&\textbf{47} & articoli di collaborazione totali, in riviste internazionali peer-reviewed\\
&\textbf{6}& \, articoli in fase preprint,\\
&\textbf{2}& \, altre pubblicazioni (tesi di dottorato, white papers, reviews)
\end{tabular} }
\textcolor{mark_color}{\textbf{Numero totale di citazioni}}: >14400.
\textcolor{mark_color}{\textbf{h-index}}: 23 (secondo record ADS e iNSPIRE).
\\
\textcolor{mark_color}{\textbf{Link a profili di citazione}}:
\href{https://ui.adsabs.harvard.edu/search/fq=%7B!type%3Daqp%20v%3D%24fq_doctype%7D&fq_doctype=(doctype%3A%22misc%22%20OR%20doctype%3A%22inproceedings%22%20OR%20doctype%3A%22article%22%20OR%20doctype%3A%22eprint%22)&q=%20author%3A%22Buscicchio%2C%20Riccardo%22&sort=citation_count%20desc%2C%20bibcode%20desc&p_=0}{\textsc{ADS}};
\href{https://inspirehep.net/literature?sort=mostrecent&size=25&page=1&q=author%3AR.Buscicchio&ui-citation-summary=true}{\textsc{iNSPIRE}};
\href{http://arxiv.org/a/buscicchio_r_1.html}{\textsc{arXiv}};
\href{https://orcid.org/0000-0002-7387-6754}{\textsc{orcid}}.

\textbf{Lista completa delle pubblicazioni} disponibile 
%mark_CVshort
a seguire e
%mark_CVshort
all'indirizzo \\
\href{http://www.riccardobuscicchio.com/publications}{\texttt{www.riccardobuscicchio.com/publications}}.

%\cvitem{}{\begin{tabular}{rcl}
\textcolor{mark_color}{\textbf{Seminari}}: &\hspace{0.3cm} &
\textbf{29} seminari a conferenze,
\textbf{10} seminari dipartimentali,
\\ & &
\end{tabular} }

\textbf{Lista completa dei seminari} disponibile
%mark_CVshort
a seguire e
%mark_CVshort
all'indirizzo \\
\href{http://www.riccardobuscicchio.com/talks}{\texttt{www.riccardobuscicchio.com/talks}}.

\section{Grant, Premi \& Riconoscimenti}

\textbf{\textcolor{black}{Premi Accademici:}}
\vspace{0.1cm}

\cvitemwithcomment{}{\hspace{0.4cm}$\circ\;$ 
{Braccini PhD Thesis Prize}, Menzione d'onore della Gravitational Wave International Committee.}{2021}
\vspace{-0.1cm}
\cvitemwithcomment{}{\hspace{0.4cm}$\circ\;$ 
{Michael Penston PhD Thesis Prize}, Secondo premio della Royal Astronomical Society.}{2021}
\vspace{-0.1cm}

\vspace{0.2cm}

\textbf{\textcolor{black}{Grants:}}
\vspace{0.1cm}


\cvitemwithcomment{}{\hspace{0.4cm}$\circ\;$ 
{EuroHPC PRACE ``LISAS-FIT'' proposal}, 100k CPUh su Leonardo BOOSTER}{2023}
\vspace{-0.1cm}

\cvitemwithcomment{}{\hspace{0.4cm}$\circ\;$ 
{Google Cloud for Researchers}, 4kEUR Google Cloud Research Credits}{2023}
\vspace{-0.1cm}

\cvitemwithcomment{}{\hspace{0.4cm}$\circ\;$ 
{CINECA ISCRA Type-C project ``LISA-MW'' proposal}, 10k CPUh presso il Centro Nazionale per HPC.}{2022}
\vspace{-0.1cm}

\cvitemwithcomment{}{\hspace{0.4cm}$\circ\;${Grant di viaggio}, Horizon 2020 AHEAD 2020 (High Energy Astrophysics)}{2021}
\vspace{-0.1cm}

\cvitemwithcomment{}{\hspace{0.4cm}$\circ\;${Grant di viaggio}, American Physical Society, DGRAV Student Travel Grant}{2020}
\vspace{-0.1cm}

\cvitemwithcomment{}{\hspace{0.4cm}$\circ\;${Grant di viaggio}, Institute of Physics Student Travel fund}{2019}
\vspace{-0.1cm}

\cvitemwithcomment{}{\hspace{0.4cm}$\circ\;${Grant di viaggio}, Royal Astronomical Society, Regno Unito.}{2018}
\vspace{-0.1cm}

\section{Supervisione}

\vspace{0.2cm}
\textbf{\textcolor{black}{Co-supervisione studenti di Dottorato:}}
\vspace{0.1cm}
\\
%
\cvitemwithcomment{}{\hspace{0.4cm}$\circ\;$ F.~Nobili, Università dell'Insubria, }{2024-2027}
\vspace{-0.1cm}
%
\cvitemwithcomment{}{\hspace{0.4cm}$\circ\;$ A.~Spadaro, Università di Milano-Bicocca.}{2022-2025}
\vspace{-0.1cm}
%
\cvitemwithcomment{}{\hspace{0.4cm}$\circ\;$ F.~Pozzoli, Università dell'Insubria.}{2022-2025}
\vspace{-0.1cm}

\vspace{0.2cm}
\textbf{\textcolor{black}{Co-supervisione studenti di Laurea magistrale:}}
\vspace{0.1cm}

\cvitemwithcomment{}{\hspace{0.4cm}$\circ\;$ L.~Viganò, Università di Milano-Bicocca, Tesi magistrale}{2024-2025}
\vspace{-0.1cm}
%
\cvitemwithcomment{}{\hspace{0.4cm}$\circ\;$ M.~Bellotti, Università di Milano-Bicocca, Tesi magistrale.}{2024-2025}
\vspace{-0.1cm}
%
\cvitemwithcomment{}{\hspace{0.4cm}$\circ\;$ D.~Chirico, Università di Milano-Bicocca, Tesi magistrale.}{2023-2024}
\vspace{-0.1cm}
%
\cvitemwithcomment{}{\hspace{0.4cm}$\circ\;$ S.~Corbo, Politecnico di Milano, Tesi magistrale.}{2023-2024}
\vspace{-0.1cm}
%
\cvitemwithcomment{}{\hspace{0.4cm}$\circ\;$ R.~Rosso, Università di Pisa, Tesi magistrale.}{2023-2024}
\vspace{-0.1cm}
%
\cvitemwithcomment{}{\hspace{0.4cm}$\circ\;$ G.~Astorino, Università di Pisa, Tesi magistrale.}{2023-2024}
\vspace{-0.1cm}
%
\cvitemwithcomment{}{\hspace{0.4cm}$\circ\;$ M.~Piarulli, Università di Milano-Bicocca, Tesi magistrale.}{2022-2023}
\vspace{-0.1cm}
\hspace{0.4cm}$\phantom{\circ}\;$(ora studente di dottorato presso Univ. di Tolosa, Francia)
\vspace{0.1cm}

%
\cvitemwithcomment{}{\hspace{0.4cm}$\circ\;$ A.~Spadaro, Università di Milano-Bicocca, Tesi magistrale.}{2021-2022}
\vspace{-0.1cm}
\hspace{0.4cm}$\phantom{\circ}\;$(ora studentessa di dottorato presso Università di Milano-Bicocca, Italia)
\vspace{0.1cm}

%
\cvitemwithcomment{}{\hspace{0.4cm}$\circ\;$ A.~Carzaniga, Università di Milano-Bicocca, Tesi magistrale.}{2021-2022}
\vspace{-0.1cm}
%
\cvitemwithcomment{}{\hspace{0.4cm}$\circ\;$ A.~Geminardi, Università di Milano-Bicocca, Tesi magistrale.}{2021-2022}
\vspace{-0.1cm}
\hspace{0.4cm}$\phantom{\circ}\;$ (ora studente di dottorato presso Univ. di Pavia, Italia)
\vspace{0.1cm}

%
\cvitemwithcomment{}{\hspace{0.4cm}$\circ\;$ E.~Finch, Università di Birmingham, Tesi magistrale.}{2018}
\vspace{-0.1cm}
%
\cvitemwithcomment{}{\hspace{0.4cm}$\circ\;$ V.~Spasova, Università di Birmingham, Tesi magistrale.}{2018}
\vspace{-0.1cm}
%

\vspace{0.2cm}
\textbf{\textcolor{black}{Co-supervisione studenti di Laurea Triennale:}}
\vspace{0.1cm}

%
\cvitemwithcomment{}{\hspace{0.4cm}$\circ\;$ H.~P.~G.~Carabajo, Università di Milano-Bicocca, Tesi triennale.}{2023-2024}
\vspace{-0.1cm}
%

\section{Insegnamenti, assistente alla didattica}

\vspace{0.2cm}
\textbf{\textcolor{black}{Insegnamenti:}}
\vspace{0.1cm}

\cvitemwithcomment{}{\hspace{0.4cm}$\circ\;$ Current and future challenges in GW astronomy, Corso di Dottorato, Milano-Bicocca (Italia).}{2023}\vspace{-0.1cm}

\cvitemwithcomment{}{\hspace{0.4cm}$\circ\;$ Mathematical physics and gravity (MAF900), Corso di dottorato, Univ.~of Stavanger (Norvegia).}{2023}\vspace{-0.1cm}

\cvitemwithcomment{}{\hspace{0.4cm}$\circ\;$ Lezioni per il corso di Astrostatistica (F5802Q014/20), Laurea Magistrale in Astrofisica}{2022}\vspace{-0.1cm}
\hspace{0.4cm}$\phantom{\circ}\;$  Univ. di Milano-Bicocca (Italia)

\vspace{0.2cm}
\textbf{\textcolor{black}{Esercitatore:}}
\vspace{0.05cm}

\cvitemwithcomment{}{\hspace{0.4cm}$\circ\;$ Python Computing Lab, Bachelor's degree in Physics, Univ. of Birmingham, Regno Unito}{2017-2021}\vspace{-0.1cm} 

\cvitemwithcomment{}{\hspace{0.4cm}$\circ\;$ Maths for physicists, Bachelor's degree in Physics, Univ. of Birmingham, Regno Unito}{2017-2019}\vspace{-0.1cm}

\cvitemwithcomment{}{\hspace{0.4cm}$\circ\;$ Physics and communication skills, Master's degree in Physics, Univ. of Birmingham, Regno Unito.}{2019}\vspace{-0.1cm}

\vspace{0.2cm}

\section{Coordinamento collaborazioni internazionali, responsabilità editoriali e di ricerca}

\textbf{\textcolor{black}{Responsabilità collaborazioni internazionali}}
\vspace{0.1cm}

\cvitemwithcomment{}{\hspace{0.4cm}Co-chair della Coordination Unit L2D (Global Fit, ESA LISA Project Office)}{2024-2025}


\textbf{\textcolor{black}{Referee per riviste scientifiche}}
\vspace{0.1cm}

\begin{tabular}{@{\hskip 0.4cm}l@{\hskip 0.4in}l}
$\circ\;$ Physical Review Letters & $\circ\;$ Physical Review D \\
$\circ\;$ The Astrophysical Journal Letters  &  $\circ\;$  The Open Journal of Astrophysics  \\
$\circ\;$ Journal of Cosmology and Astroparticle Physics & $\circ\;$ NASA Technology Transfer Program \\
$\circ\;$ Monthly Notices of the Royal Astronomical Society& $\circ\;$ Classical and Quantum Gravity\\
$\circ\;$ Institute of Physics Trusted Reviewer Excellence program \\
\end{tabular}

\vspace{0.2cm}
\textbf{\textcolor{black}{Responsabilità editoriali}}
\vspace{0.1cm}

%\cvitemwithcomment{}{\hspace{0.4cm}$\circ\;$ MDPI Universe}{2024}\vspace{-0.1cm}
%\hspace{0.4cm}$\phantom{\circ}\;$ Special Issue \textit{``Challenges and Synergies with Future Gravitational Wave Observatories''.}\vspace{0.1cm}

\cvitemwithcomment{}{\hspace{0.4cm}$\circ\;$ Board editorial per il Lensing Working Group nella collaborazione LIGO,Virgo, KAGRA}{2023}\vspace{-0.1cm}

\cvitemwithcomment{}{\hspace{0.4cm}$\circ\;$ Co-editor dell'issue \text{``LISA data analysis''} per Living Review in Relativity}{2022-2023}\vspace{-0.1cm}

\vspace{0.2cm}
\textbf{\textcolor{black}{Organizzazione di conferenze e workshop}}
\vspace{0.1cm}

\cvitemwithcomment{}{\hspace{0.4cm}$\circ\;$ \textit{LISA Distributed Data Processing Center June Workshop}, Milano, Italia.}{2025}\vspace{-0.1cm}

\cvitemwithcomment{}{\hspace{0.4cm}$\circ\;$ \href{https://www.ifpu.it/}{IFPU focus week} on \href{https://sites.google.com/unimib.it/gwemerge/}{``\textit{Emerging methods in GW population inference}``}, Trieste, Italia.}{2024}\vspace{-0.1cm}

\cvitemwithcomment{}{\hspace{0.4cm}$\circ\;$ \textit{LISA Astrophysics Working Group Conference}, Birmingham, Regno Unito.}{2022}\vspace{-0.1cm}

\cvitemwithcomment{}{\hspace{0.4cm}$\circ\;$ \textit{Gravitational-wave populations: what's next?}, Milano, Italia.}{2023}\vspace{-0.1cm}

\cvitemwithcomment{}{\hspace{0.4cm}$\circ\;$ \textit{Gravitational-wave Excellence Alliance Training (GrEAT) PhD school}, Birmingham, Regno Unito.}{2019}\vspace{-0.1cm}

\cvitemwithcomment{}{\hspace{0.4cm}$\circ\;$ \textit{Gravitational-wave Open Science Center First Open Data Workshop}, (online)}{2019}\vspace{-0.1cm}

%mark_CVshort
\vspace{0.2cm}
\textbf{\textcolor{black}{Divulgazione \& Terza missione}}
\vspace{0.1cm}

\cvitemwithcomment{}{\hspace{0.4cm}$\circ\;$ Orientamento a studenti di scuole superiori nell'ambito}{2024-2025}\vspace{-0.1cm}
\hspace{0.4cm}$\phantom{\circ}\;$ del progetto \textit{Orientamento PNRR}, Milano, Italia.

\cvitemwithcomment{}{\hspace{0.4cm}$\circ\;$ Sviluppo di illustrazioni e animazioni per il LISA Consortium}{2023}\vspace{-0.1cm}

\cvitemwithcomment{}{\hspace{0.4cm}$\circ\;$ Sviluppo di illustrazioni e contenuti per LIGO Magazine}{2022-2023}\vspace{-0.1cm}

\cvitemwithcomment{}{\hspace{0.4cm}$\circ\;$ Sviluppo di una interfaccia web e app per la visualizzazione di mappe}{2022-2023}\vspace{-0.1cm}
\hspace{0.4cm}$\phantom{\circ}\;$ di localizzazione da alerte GW. \href{https://chirp.sr.bham.ac.uk/}{\texttt{https://chirp.sr.bham.ac.uk}}

\cvitemwithcomment{}{\hspace{0.4cm}$\circ\;$ Organizzazione di eventi di bisettimanali di divulgazione}{2017-2021}\vspace{-0.1cm}
\hspace{0.4cm}$\phantom{\circ}\;$ \textit{``Astronomy in the city''}, Birmingham, Regno Unito.

\cvitemwithcomment{}{\hspace{0.4cm}$\circ\;$ Organizzatore di ciclo di incontri ``PhD meet and greet'', Università di Birmingham}{2021}\vspace{-0.1cm}

\cvitemwithcomment{}{\hspace{0.4cm}$\circ\;$ Lezioni a studenti di scuole medie superiori, Italia}{dal 2021}\vspace{-0.1cm}
%mark_CVshort

\vspace{0.2cm}
\textbf{\textcolor{black}{Riconoscimenti, qualifiche e cariche accademiche}}
\vspace{0.1cm}


\cvitemwithcomment{}{\hspace{0.4cm}$\circ\;$  Abilitazione Scientifica Nazionale a Professore di Seconda Fascia}{2023}
\hspace{0.4cm}$\phantom{\circ}\;$ (Settore 02/C1, GSD 02/PHYS-05 - SSD PHYS-05/A).

\cvitemwithcomment{}{\hspace{0.4cm}$\circ\;$ Abilitazione all'insegnamento accademico in Astrofisica (Sec.34).}{2023}\vspace{-0.1cm}
\hspace{0.4cm}$\phantom{\circ}\;$ Ministero francese dell'istruzione e della ricerca (qualificazione no.23234388826).

%mark_CVshort
\cvitemwithcomment{}{\hspace{0.4cm}$\circ\;$  Segretario della Consulta degli Assegnisti, Università di Milano-Bicocca}{2024-2025}\vspace{-0.1cm}
\cvitemwithcomment{}{\hspace{0.4cm}$\circ\;$  Rappresentante degli Assegnisti, Dipartimento di Fisica, Università di Milano-Bicocca}{2023-2025}\vspace{-0.1cm}
\cvitemwithcomment{}{\hspace{0.4cm}$\circ\;$ LIGO Science Collaboration Academic Advisory Committee.}{2019-2021}\vspace{-0.1cm}
%mark_CVshort

\vspace{0.2cm}
\textbf{\textcolor{black}{Affiliazioni accademiche}}
\vspace{0.1cm}

\cvitemwithcomment{}{\hspace{0.4cm}$\circ\;$ LISA Consortium, core member.}{dal 2018}\vspace{-0.1cm}

\cvitemwithcomment{}{\hspace{0.4cm}$\circ\;$ Italian Center for Supercomputing (ICSC).}{dal 2021}\vspace{-0.1cm}

\cvitemwithcomment{}{\hspace{0.4cm}$\circ\;$ TEONGRAV National Initiative (Gravity Theory)}{dal 2021}\vspace{-0.1cm} 
\hspace{0.4cm}$\phantom{\circ}\;$ Italian National Institute for Nuclear Physics (INFN).

\cvitemwithcomment{}{\hspace{0.4cm}$\circ\;$ LIGO, Virgo, Kagra Collaboration, full member}{dal 2017}\vspace{-0.1cm}

\cvitemwithcomment{}{\hspace{0.4cm}$\circ\;$ Società Italiana di Relatività Generale e Fisica Gravitazionale (SIGRAV)}{dal 2021}\vspace{-0.1cm}

\cvitemwithcomment{}{\hspace{0.4cm}$\circ\;$ Istituto Nazionale di Astrofisica (INAF)}{dal 2021}\vspace{-0.1cm}

\cvitemwithcomment{}{\hspace{0.4cm}$\circ\;$ American Physical Society (APS)}{}\vspace{-0.1cm}

\cvitemwithcomment{}{\hspace{0.4cm}$\circ\;$ Società Italiana di Fisica (SIF)}{2021}\vspace{-0.1cm}

\cvitemwithcomment{}{\hspace{0.4cm}$\circ\;$ Royal Astronomical Society (RAS) fellow.}{2018-2021}\vspace{-0.1cm}

%mark_CVshort
\section{Competenze}
\cvitem{Linguaggi di programmazione}{Python (avanzato), Bash (avanzato), Julia (avanzato), Mathematica, Go, R (avanzato), Stan, C, Qt5.}
\cvitem{Altri strumenti di programmazione}{TensorFlow, LIGO lalsuite, \LaTeX, controllo versione, strumenti HPC, containerization, continuous integration, cloud computing, sviluppo web.}
\cvitem{Lingue}{Inglese (C2), Italiano (madrelingua), Francese (A2)}
%mark_CVshort

%mark_CVshort
\section{Hobbies}
Nuoto, corsa, arrampicata sportiva, fotografia. Fantascienza, musica elettronica.
%mark_CVshort

%mark_CVshort
%\pagebreak
%\section{Lista completa di pubblicazioni}\vspace{0.2cm} 

%\textcolor{color1}{\textbf{Submitted short-author and collaboration papers which I have substantially contributed to.:}}
\vspace{-0.5cm}

\cvitem{}{\small\hspace{-1cm}\begin{longtable}{rp{0.3cm}p{15.8cm}}
%
\textbf{5.} & & \textit{Functional inference on deviations from General Relativity.}
\newline{}
C. Pacilio, \textbf{R. Buscicchio}.
\newline{}
\href{https://arxiv.org/abs/2507.13454[gr-qc]}{arXiv:2507.13454[gr-qc].}
\vspace{0.09cm}\\
%
\textbf{4.} & & \textit{Comparing astrophysical models to gravitational-wave data in the observable space.}
\newline{}
A. Toubiana, D. Gerosa, M. Mould, S. Rinaldi, M. Arca Sedda, T. Bruel, \textbf{R. Buscicchio}, J. Gair, L. Paiella, F. Santoliquido, R. Tenorio, C. Ugolini.
\newline{}
\href{https://arxiv.org/abs/2507.13249[gr-qc]}{arXiv:2507.13249[gr-qc].}
\vspace{0.09cm}\\
%
\textbf{3.} & & \textit{Bahamas: BAyesian inference with HAmiltonian Montecarlo for Astrophysical Stochastic background.}
\newline{}
F. Pozzoli, \textbf{R. Buscicchio}, A. Klein, D. Chirico.
\newline{}
\href{https://arxiv.org/abs/2506.22542[astro-ph.IM]}{arXiv:2506.22542[astro-ph.IM].}
\vspace{0.09cm}\\
%
\textbf{2.} & & \textit{LISA Definition Study Report.}
\newline{}
M. Colpi, K. Danzmann, M. Hewitson, K. Holley-Bockelmann, et al. (incl. \textbf{R. Buscicchio}).
\newline{}
\href{https://arxiv.org/abs/2402.07571}{arXiv:2402.07571 [astro-ph.CO].}
\vspace{0.09cm}\\
%
\textbf{1.} & & \textit{The last three years: multiband gravitational-wave observations of stellar-mass binary black holes.}
\newline{}
A. Klein, G. Pratten, \textbf{R. Buscicchio}, P. Schmidt, C. J. Moore, E. Finch, A. Bonino, L. M. Thomas, N. Williams, D. Gerosa, S. McGee, M. Nicholl, A. Vecchio.
\newline{}
\href{https://arxiv.org/abs/2204.03423}{arXiv:2204.03423 [astro-ph.HE].}
\vspace{0.09cm}\\
%
\end{longtable} }
\textcolor{color1}{\textbf{Short-author papers in major peer-reviewed journals:}}
\vspace{-0.5cm}

\cvitem{}{\small\hspace{-1cm}\begin{longtable}{rp{0.3cm}p{15.8cm}}
%
\textbf{32.} & & \textit{Environmental effects in the LISA stochastic signal from stellar-mass black hole binaries.}
\newline{}
R. Chen, R. S. Chandramouli, F. Pozzoli, \textbf{R. Buscicchio}, E. Barausse.
\newline{}
\href{https://doi.org/10.1103/w61d-3jk5}{\prd 112, (2025), (in press)}. \href{https://arxiv.org/abs/2507.00694[gr-qc]}{arXiv:2507.00694[gr-qc].}
\vspace{0.09cm}\\
%
\textbf{31.} & & \textit{Variability in the massive black hole binary candidate SDSS J2320+0024: no evidence for periodic modulation.}
\newline{}
F. Rigamonti, L. Bertassi, \textbf{R. Buscicchio}, F. Cocchiararo, S. Covino, M. Dotti, A. Sesana, P. Severgnini.
\newline{}
\href{https://doi.org/10.1051/0004-6361/202555550}{\aap (2025), (in press)}. \href{https://arxiv.org/abs/2505.22706[astro-ph.GA]}{arXiv:2505.22706[astro-ph.GA].}
\vspace{0.09cm}\\
%
\textbf{30.} & & \textit{Is your stochastic signal really detectable?.}
\newline{}
F. Pozzoli, J. Gair, \textbf{R. Buscicchio}, L. Speri.
\newline{}
\href{https://doi.org/10.1103/22h4-tqh9}{\prd 112, (2025) 064035}. \href{https://arxiv.org/abs/2412.10468}{arXiv:2412.10468 [astro-ph.IM].}
\vspace{0.09cm}\\
%
\textbf{29.} & & \textit{A test for LISA foreground Gaussianity and stationarity. I. Galactic white-dwarf binaries.}
\newline{}
\textbf{R. Buscicchio}, A. Klein, V. Korol, F. Di Renzo, C.J. Moore, D. Gerosa, A. Carzaniga.
\newline{}
\href{https://doi.org/10.1140/epjc/s10052-025-14616-w}{\epjc 85, (2025) 887}. \href{https://arxiv.org/abs/2410.08263}{arXiv:2410.08263 [astro-ph.HE].}
\vspace{0.09cm}\\
%
\textbf{28.} & & \textit{Accelerating LISA inference with Gaussian processes.}
\newline{}
J. El Gammal, \textbf{R. Buscicchio}, G. Nardini, J. Torrado.
\newline{}
\href{https://doi.org/10.1103/c66v-rl3w}{\prd 112, (2025) 063010}. \href{https://arxiv.org/abs/2503.21871}{arXiv:2503.21871 [astro-ph.HE].}
\vspace{0.09cm}\\
%
\textbf{27.} & & \textit{Test for LISA foreground Gaussianity and stationarity: extreme mass-ratio inspirals.}
\newline{}
M. Piarulli, \textbf{R. Buscicchio}, F. Pozzoli, O. Burke, M. Bonetti, A. Sesana.
\newline{}
\href{https://doi.org/10.1103/nfn4-pgr5}{\prd 111, (2025) 103047}. \href{https://arxiv.org/abs/2410.08862}{arXiv:2410.08862 [astro-ph.HE].}
\vspace{0.09cm}\\
%
\textbf{26.} & & \textit{Cyclostationary signals in LISA: a practical application to Milky Way satellites.}
\newline{}
F. Pozzoli, \textbf{R. Buscicchio}, A. Klein, V. Korol, A. Sesana, F. Haardt.
\newline{}
\href{https://doi.org/10.1103/PhysRevD.111.063005}{\prd 111, (2025) 063005}. \href{https://arxiv.org/abs/2410.08274}{arXiv:2410.08274 [astro-ph.GA].}
\vspace{0.09cm}\\
%
\textbf{25.} & & \textit{Characterization of non-Gaussian stochastic signals with heavier-tailed likelihoods.}
\newline{}
N. Karnesis, A. Sasli, \textbf{R. Buscicchio}, N. Stergioulas.
\newline{}
\href{https://doi.org/10.1103/PhysRevD.111.022005}{\prd 111, (2025) 022005}. \href{https://arxiv.org/abs/2410.14354}{arXiv:2410.14354 [gr-qc].}
\vspace{0.09cm}\\
%
\textbf{24.} & & \textit{Stellar-mass black-hole binaries in LISA: characteristics and complementarity with current-generation interferometers.}
\newline{}
\textbf{R. Buscicchio}, J. Torrado, C. Caprini, G. Nardini, M. Pieroni, N. Karnesis, A. Sesana.
\newline{}
\href{https://doi.org/10.1088/1475-7516/2025/01/084}{\jcap 01 (2025) 084}. \href{https://arxiv.org/abs/2410.18171}{arXiv:2410.18171 [astro-ph.HE].}
\vspace{0.09cm}\\
%
\textbf{23.} & & \textit{Stars or gas? Constraining the hardening processes of massive black-hole binaries with LISA.}
\newline{}
A. Spadaro, \textbf{R. Buscicchio}, D. Izquierdo--Villalba, D. Gerosa, A. Klein, G. Pratten.
\newline{}
\href{https://doi.org/10.1103/PhysRevD.111.023004}{\prd 111, (2025) 023004}. \href{https://arxiv.org/abs/2409.13011}{arXiv:2409.13011 [astro-ph.HE].}
\vspace{0.09cm}\\
%
\textbf{22.} & & \textit{Partial alignment between jets and megamasers: coherent or selective accretion?.}
\newline{}
M. Dotti, \textbf{R. Buscicchio}, F. Bollati, R. Decarli, W. Del Pozzo, A. Franchini.
\newline{}
\href{https://doi.org/10.1051/0004-6361/202450112}{\aap 692 (2024) A233}. \href{https://arxiv.org/abs/2403.18002}{arXiv:2403.18002 [astro-ph.GA].}
\vspace{0.09cm}\\
%
\textbf{21.} & & \textit{Expected insights on type Ia supernovae from LISA's gravitational wave observations.}
\newline{}
V. Korol, \textbf{R. Buscicchio}, Ruediger Pakmor, Javier Morán-Fraile, Christopher J. Moore, Selma E. de Mink.
\newline{}
\href{https://www.aanda.org/articles/aa/full_html/2024/11/aa51380-24/aa51380-24.html}{\aap 691 (2024) A44}. \href{https://arxiv.org/abs/2407.03935}{arXiv:2407.03935 [astro-ph.HE].}
\vspace{0.09cm}\\
%
\textbf{20.} & & \textit{A weakly-parametric approach to stochastic background inference in LISA.}
\newline{}
F. Pozzoli, \textbf{R. Buscicchio}, C. J. Moore, A. Sesana, F. Haardt, A. Sesana.
\newline{}
\href{https://journals.aps.org/prd/abstract/10.1103/PhysRevD.109.083029}{\prd 109, (2024) 083029}. \href{https://arxiv.org/abs/2311.12111}{arXiv:2311.12111 [astro-ph.CO].}
\vspace{0.09cm}\\
%
\textbf{19.} & & \textit{A fast test for the identification and confirmation of massive black hole binary.}
\newline{}
M. Dotti, F. Rigamonti, S. Rinaldi, W. Del Pozzo, R. Decarli, \textbf{R. Buscicchio}.
\newline{}
\href{https://www.aanda.org/articles/aa/abs/2023/12/aa46916-23/aa46916-23.html}{\aap 680 (2023) A69}. \href{https://arxiv.org/abs/2310.06896}{arXiv:2310.06896 [astro-ph.HE].}
\vspace{0.09cm}\\
%
\textbf{18.} & & \textit{Glitch systematics on the observation of massive black-hole binaries with LISA.}
\newline{}
A. Spadaro, \textbf{R. Buscicchio}, D. Vetrugno, A. Klein, D. Gerosa, S. Vitale, R. Dolesi, W. J. Weber, M. Colpi.
\newline{}
\href{https://journals.aps.org/prd/abstract/10.1103/PhysRevD.108.123029}{\prd 108 (2023) 123029}. \href{https://arxiv.org/abs/2306.03923}{arXiv:2306.03923 [gr-qc].}
\vspace{0.09cm}\\
%
\textbf{17.} & & \textit{Implications of pulsar timing array observations for LISA detections of massive black hole binaries.}
\newline{}
N. Steinle, H. Middleton, C. J. Moore, S. Chen, A. Klein, G. Pratten, \textbf{R. Buscicchio}, E. Finch, A. Vecchio.
\newline{}
\href{https://academic.oup.com/mnras/article/525/2/2851/7244712}{\mnras 525 2 (2023)}. \href{https://arxiv.org/abs/2305.05955}{arXiv:2305.05955 [astro-ph.HE].}
\vspace{0.09cm}\\
%
\textbf{16.} & & \textit{Parameter estimation of binary black holes in the endpoint of the up-down instability.}
\newline{}
V. De Renzis, D. Gerosa, M. Mould, \textbf{R. Buscicchio}, L. Zanga.
\newline{}
\href{https://journals.aps.org/prd/abstract/10.1103/PhysRevD.108.024024}{\prd 108 (2023) 024024}. \href{https://arxiv.org/abs/2304.13063}{arXiv:2304.13063 [gr-qc].}
\vspace{0.09cm}\\
%
\textbf{15.} & & \textit{Improved detection statistics for non Gaussian gravitational wave stochastic backgrounds.}
\newline{}
M. Ballelli, \textbf{R. Buscicchio}, B. Patricelli, A. Ain, G. Cella.
\newline{}
\href{https://journals.aps.org/prd/abstract/10.1103/PhysRevD.107.124044}{\prd 107 (2023) 124044}. \href{https://arxiv.org/abs/2212.10038}{arXiv:2212.10038 [gr-qc].}
\vspace{0.09cm}\\
%
\textbf{14.} & & \textit{Detecting non-Gaussian gravitational wave backgrounds: a unified framework.}
\newline{}
\textbf{R. Buscicchio}, A. Ain, M. Ballelli, G. Cella, B. Patricelli.
\newline{}
\href{https://journals.aps.org/prd/abstract/10.1103/PhysRevD.107.063027}{\prd 107 (2023) 063027}. \href{https://arxiv.org/abs/2209.01400}{arXiv:2209.01400 [gr-qc].}
\vspace{0.09cm}\\
%
\textbf{13.} & & \textit{Detectability of a spatial correlation between stellar-mass black hole mergers and Active Galactic Nuclei in the Local Universe.}
\newline{}
N. Veronesi, E.M. Rossi, S. van Velzen, \textbf{R. Buscicchio}.
\newline{}
\href{https://academic.oup.com/mnras/article/514/2/2092/6587069}{\mnras 514 2 (2023)}. \href{https://arxiv.org/abs/2203.05907}{arXiv:2203.05907 [astro-ph.HE].}
\vspace{0.09cm}\\
%
\textbf{12.} & & \textit{Bayesian parameter estimation of stellar-mass black-hole binaries with LISA.}
\newline{}
\textbf{R. Buscicchio}, A. Klein, E. Roebber, C. J. Moore, D. Gerosa, E. Finch, A. Vecchio.
\newline{}
\href{https://journals.aps.org/prd/abstract/10.1103/PhysRevD.104.044065}{\prd 104 (2021) 044065}. \href{https://arxiv.org/abs/2106.05259}{arXiv:2106.05259 [astro-ph.HE].}
\vspace{0.09cm}\\
%
\textbf{11.} & & \textit{An Interactive Gravitational-Wave Detector Model for Museums and Fairs.}
\newline{}
S. J. Cooper, A. C. Green, H. R. Middleton, C. P. L. Berry, \textbf{R. Buscicchio}, E. Butler, C. J. Collins, C. Gettings, D. Hoyland, A. W. Jones, J. H. Lindon, I. Romero-Shaw, S. P. Stevenson, E. P. Takeva, S. Vinciguerra, A. Vecchio, C. M. Mow-Lowry, A. Freise.
\newline{}
\href{https://pubs.aip.org/aapt/ajp/article/89/7/702/1056907/An-interactive-gravitational-wave-detector-model}{\ajp 89 (2021) 702–712}. \href{https://arxiv.org/abs/2004.03052}{arXiv:2004.03052 [physics.ed-ph].}
\vspace{0.09cm}\\
%
\textbf{10.} & & \textit{Evidence for hierarchical black hole mergers in the second LIGO--Virgo gravitational-wave catalog.}
\newline{}
C. Kimball, C. Talbot, C.P.L. Berry, M. Zevin, E. Thrane, V. Kalogera, \textbf{R. Buscicchio}, M. Carney, T. Dent, H. Middleton, E. Payne, J. Veitch, D. Williams .
\newline{}
\href{https://iopscience.iop.org/article/10.3847/2041-8213/ac0aef}{\apjl 915 (2021) L35}. \href{https://arxiv.org/abs/2011.05332}{arXiv:2011.05332 [astro-ph.HE].}
\vspace{0.09cm}\\
%
\textbf{9.} & & \textit{Testing general relativity with gravitational-wave catalogs: the insidious nature of waveform systematics.}
\newline{}
C. J. Moore, E. Finch, \textbf{R. Buscicchio}, D. Gerosa.
\newline{}
\href{https://www.sciencedirect.com/science/article/pii/S2589004221005459}{iScience 24 (2021) 102577}. \href{https://arxiv.org/abs/2103.16486}{arXiv:2103.16486   [gr-qc].}
\vspace{0.09cm}\\
%
\textbf{8.} & & \textit{LoCuSS: The splashback radius of massive galaxy clusters and its dependence on cluster merger history.}
\newline{}
M. Bianconi, \textbf{R. Buscicchio}, G. P. Smith, S. L. McGee, C.P. Haines, A. Finoguenov, A. Babul.
\newline{}
\href{https://iopscience.iop.org/article/10.3847/1538-4357/abebd7}{\apj 911 (2021) 136}. \href{https://arxiv.org/abs/2010.05920}{arXiv:2010.05920 [astro-ph.GA].}
\vspace{0.09cm}\\
%
\textbf{7.} & & \textit{Search for Black Hole Merger Families.}
\newline{}
D. Veske, A. G. Sullivan, Z. Marka, I. Bartos, K. R. Corley, J. Samsing, \textbf{R. Buscicchio}, S. Marka.
\newline{}
\href{https://iopscience.iop.org/article/10.3847/2041-8213/abd721}{\apjl 907 (2021) L48}. \href{https://arxiv.org/abs/2011.06591}{arXiv:2011.06591 [astro-ph.HE].}
\vspace{0.09cm}\\
%
\textbf{6.} & & \textit{Constraining the lensing of binary black holes from their stochastic background.}
\newline{}
\textbf{R. Buscicchio}, C. J. Moore, G. Pratten, P. Schmidt, M. Bianconi, A. Vecchio.
\newline{}
\href{https://journals.aps.org/prl/abstract/10.1103/PhysRevLett.125.141102}{\prl 125 (2020) 141102}. \href{https://arxiv.org/abs/2006.04516}{arXiv:2006.04516 [astro-ph.CO].}
\vspace{0.09cm}\\
%
\textbf{5.} & & \textit{Constraining the lensing of binary neutron stars from their stochastic background.}
\newline{}
\textbf{R. Buscicchio}, C. J. Moore, G. Pratten, P. Schmidt, A. Vecchio.
\newline{}
\href{https://journals.aps.org/prd/abstract/10.1103/PhysRevD.102.081501}{\prd 102 (2020) 081501 }. \href{https://arxiv.org/abs/2008.12621}{arXiv:2008.12621 [astro-ph.HE].}
\vspace{0.09cm}\\
%
\textbf{4.} & & \textit{Measuring precession in asymmetric compact binaries.}
\newline{}
G. Pratten, P. Schmidt, \textbf{R. Buscicchio}, L. M. Thomas.
\newline{}
\href{https://journals.aps.org/prresearch/abstract/10.1103/PhysRevResearch.2.043096}{\prr 2 (2020) 043096}. \href{https://arxiv.org/abs/2006.16153}{arXiv:2006.16153 [gr-qc].}
\vspace{0.09cm}\\
%
\textbf{3.} & & \textit{Populations of double white dwarfs in Milky Way satellites and their detectability with LISA.}
\newline{}
V. Korol, S. Toonen, A. Klein, V. Belokurov, F. Vincenzo, \textbf{R. Buscicchio}, D. Gerosa, C. J. Moore, E. Roebber, E. M. Rossi, A. Vecchio.
\newline{}
\href{https://www.aanda.org/articles/aa/abs/2020/06/aa37764-20/aa37764-20.html}{\aap 638 (2020) A153}. \href{https://arxiv.org/abs/2002.10462}{arXiv:2002.10462 [astro-ph.GA].}
\vspace{0.09cm}\\
%
\textbf{2.} & & \textit{Milky Way satellites shining bright in gravitational waves.}
\newline{}
E. Roebber, \textbf{R. Buscicchio}, A. Vecchio, C. J. Moore, A. Klein, V. Korol, S. Toonen, D. Gerosa, J. Goldstein, S. M. Gaebel, T. E. Woods.
\newline{}
\href{https://iopscience.iop.org/article/10.3847/2041-8213/ab8ac9}{\apjl 894 (2020) L15}. \href{https://arxiv.org/abs/2002.10465}{arXiv:2002.10465 [astro-ph.GA].}
\vspace{0.09cm}\\
%
\textbf{1.} & & \textit{Label Switching Problem in Bayesian Analysis for Gravitational Wave Astronomy.}
\newline{}
\textbf{R. Buscicchio}, E. Roebber, J. M. Goldstein, C. J. Moore .
\newline{}
\href{https://journals.aps.org/prd/abstract/10.1103/PhysRevD.100.084041}{\prd 100 (2019) 084041}. \href{https://arxiv.org/abs/1907.11631}{arXiv:1907.11631 [astro-ph.IM].}
\vspace{0.09cm}\\
%
\end{longtable} }
\textcolor{color1}{\textbf{Collaboration papers in major peer-reviewed journals, which I have substantially contributed to.:}}
\vspace{-0.5cm}

\cvitem{}{\small\hspace{-1cm}\begin{longtable}{rp{0.3cm}p{15.8cm}}
%
\textbf{13.} & & \textit{Search for gravitational-lensing signatures in the full third observing run of the LIGO-Virgo network.}
\newline{}
LIGO Scientific Collaboration, Virgo Collaboration, KAGRA collaboration.
\newline{}
\href{https://iopscience.iop.org/article/10.3847/1538-4357/ad3e83}{\apj 970 (2021) 191}. \href{https://arxiv.org/abs/2304.08393}{arXiv:2304.08393 [gr-qc].}
\vspace{0.09cm}\\
%
\textbf{12.} & & \textit{GWTC-2.1: Deep Extended Catalog of Compact Binary Coalescences Observed by LIGO and Virgo During the First Half of the Third Observing Run.}
\newline{}
LIGO Scientific Collaboration, Virgo Collaboration, KAGRA collaboration.
\newline{}
\href{https://journals.aps.org/prd/abstract/10.1103/PhysRevD.109.022001}{\prd 109 (2024) 022001}. \href{https://arxiv.org/abs/2108.01045}{arXiv:2108.01045 [gr-qc].}
\vspace{0.09cm}\\
%
\textbf{11.} & & \textit{The population of merging compact binaries inferred using gravitational waves through GWTC-3.}
\newline{}
LIGO Scientific Collaboration, Virgo Collaboration, KAGRA collaboration.
\newline{}
\href{https://journals.aps.org/prx/abstract/10.1103/PhysRevX.13.011048}{\prx 13 (2021) 011048}. \href{https://arxiv.org/abs/2111.03634}{arXiv:2111.03634 [astro-ph.HE].}
\vspace{0.09cm}\\
%
\textbf{10.} & & \textit{Tests of General Relativity with GWTC-3.}
\newline{}
LIGO Scientific Collaboration, Virgo Collaboration, KAGRA collaboration.
\newline{}
\href{https://journals.aps.org/prd/accepted/17075Qf4Z7b11729787e85f1c18faca230d51e013}{\prd (in press)}. \href{https://arxiv.org/abs/2112.06861}{arXiv:2112.06861 [gr-qc].}
\vspace{0.09cm}\\
%
\textbf{9.} & & \textit{Search for lensing signatures in the gravitational-wave observations from the first half of LIGO-Virgo's third observing run.}
\newline{}
LIGO Scientific Collaboration, Virgo Collaboration, KAGRA collaboration.
\newline{}
\href{https://iopscience.iop.org/article/10.3847/1538-4357/ac23db}{\apjl (2021) 923}. \href{https://arxiv.org/abs/2105.06384}{arXiv:2105.06384 [gr-qc].}
\vspace{0.09cm}\\
%
\textbf{8.} & & \textit{GWTC-3: Compact Binary Coalescences Observed by LIGO and Virgo During the Second Part of the Third Observing Run.}
\newline{}
LIGO Scientific Collaboration, Virgo Collaboration, KAGRA collaboration.
\newline{}
\href{https://journals.aps.org/prx/abstract/10.1103/PhysRevX.13.041039}{\prx 13 (2023) 041039}. \href{https://arxiv.org/abs/2111.03606}{arXiv:2111.03606 [gr-qc].}
\vspace{0.09cm}\\
%
\textbf{7.} & & \textit{Observation of gravitational waves from two neutron star-black hole coalescences.}
\newline{}
LIGO Scientific Collaboration, Virgo Collaboration.
\newline{}
\href{https://iopscience.iop.org/article/10.3847/2041-8213/ac082e}{\apjl, 915, L5 (2021)}. \href{https://arxiv.org/abs/2106.15163}{arXiv:2106.15163 [astro-ph.HE].}
\vspace{0.09cm}\\
%
\textbf{6.} & & \textit{GWTC-2: Compact Binary Coalescences Observed by LIGO and Virgo During the First Half of the Third Observing Run.}
\newline{}
LIGO Scientific Collaboration, Virgo Collaboration.
\newline{}
\href{https://journals.aps.org/prx/abstract/10.1103/PhysRevX.11.021053}{\prx 11 (2021) 021053}. \href{https://arxiv.org/abs/2010.14527}{arXiv:2010.14527 [gr-qc].}
\vspace{0.09cm}\\
%
\textbf{5.} & & \textit{Population Properties of Compact Objects from the Second LIGO-Virgo Gravitational-Wave Transient Catalog.}
\newline{}
LIGO Scientific Collaboration, Virgo Collaboration.
\newline{}
\href{https://iopscience.iop.org/article/10.3847/2041-8213/abe949}{\apjl 913 (2021) L7}. \href{https://arxiv.org/abs/2010.14533}{arXiv:2010.14533 [astro-ph.HE].}
\vspace{0.09cm}\\
%
\textbf{4.} & & \textit{Upper Limits on the Isotropic Gravitational-Wave Background from Advanced LIGO's and Advanced Virgo's Third Observing Run.}
\newline{}
LIGO Scientific Collaboration, Virgo Collaboration, KAGRA collaboration.
\newline{}
\href{https://journals.aps.org/prd/abstract/10.1103/PhysRevD.104.022004}{\prd 104 (2021) 022004}. \href{https://arxiv.org/abs/2101.12130}{arXiv:2101.12130 [gr-qc].}
\vspace{0.09cm}\\
%
\textbf{3.} & & \textit{Binary Black Hole Population Properties Inferred from the First and Second Observing Runs of Advanced LIGO and Advanced Virgo .}
\newline{}
LIGO Scientific Collaboration, Virgo Collaboration.
\newline{}
\href{https://iopscience.iop.org/article/10.3847/2041-8213/ab3800}{\apj 882 (2019)  L24}. \href{https://arxiv.org/abs/1811.12940}{arXiv:1811.12940 [astro-ph.HE].}
\vspace{0.09cm}\\
%
\textbf{2.} & & \textit{Properties and astrophysical implications of the 150 Msun binary black hole merger GW190521.}
\newline{}
LIGO Scientific Collaboration, Virgo Collaboration.
\newline{}
\href{https://iopscience.iop.org/article/10.3847/2041-8213/aba493}{\apjl 900 (2020) L13}. \href{https://arxiv.org/abs/2009.01190}{arXiv:2009.01190 [astro-ph.HE].}
\vspace{0.09cm}\\
%
\textbf{1.} & & \textit{GW190521: A Binary Black Hole Merger with a Total Mass of 150 $M_\odot$.}
\newline{}
LIGO Scientific Collaboration, Virgo Collaboration.
\newline{}
\href{https://journals.aps.org/prl/abstract/10.1103/PhysRevLett.125.101102}{\prl 125 (2020) 101102}. \href{https://arxiv.org/abs/2009.01075}{arXiv:2009.01075 [gr-qc].}
\vspace{0.09cm}\\
%
\end{longtable} }
\textcolor{color1}{\textbf{PhD thesis, technical reports.:}}
\vspace{-0.5cm}

\cvitem{}{\small\hspace{-1cm}\begin{longtable}{rp{0.3cm}p{15.8cm}}
%
\textbf{2.} & & \textit{LISA - Laser Interferometer Space Antenna - Definition Study Report.}
\newline{}
The European Space Agency.
\newline{}
\href{https://www.cosmos.esa.int/documents/15452792/15452811/LISA_DEFINITION_STUDY_REPORT_ESA-SCI-DIR-RP-002_Public+(1).pdf/2deb7646-dccd-ae0d-75c1-b2e16df584cf?t=1707166191449}{ESA-SCI-DIR-RP-002}. 
\vspace{0.09cm}\\
%
\textbf{1.} & & \textit{Topics in Bayesian population inference for gravitational wave astronomy.}
\newline{}
\textbf{R. Buscicchio}.
\newline{}
\href{https://etheses.bham.ac.uk//id/eprint/12288/}{PhD thesis}. 
\vspace{0.09cm}\\
%
\end{longtable} }

\pagebreak
\section{Lista completa di seminari}\vspace{0.2cm} 

Invited talks marked with *.
\vspace{0.2cm}

\textcolor{color1}{\textbf{Talks at conferences:}}
\vspace{-0.5cm}

\cvitem{}{\small\hspace{-1cm}\begin{longtable}{rp{0.3cm}p{15.8cm}}
%
\textbf{29.} & * & \textit{Emergence of Milky Way structure in the first year of LISA data.}
\newline{}
CERN UniGe Gravitational Wave meeting, Geneva, Switzerland, 2025/05/23.
\vspace{0.05cm}\\
%
\textbf{28.} &  & \textit{LISA stellar-mass black holes informed by the GWTC-3 population: event rates and parameters reconstruction.}
\newline{}
LISA Astrophysics Working Group Meeting 2024, Garching, Germany, 2024/11/05.
\vspace{0.05cm}\\
%
\textbf{27.} & * & \textit{Astrophysics panel session.}
\newline{}
GRASP: Gravity Shape Pisa 2024, Pisa, Italy, 2024/10/24.
\vspace{0.05cm}\\
%
\textbf{26.} & * & \textit{Beyond Gauss? A more accurate model for LISA astrophysical noise sources.}
\newline{}
Kavli Institute for Cosmology Seminars, Cambridge, United Kingdom, 2024/10/14.
\vspace{0.05cm}\\
%
\textbf{25.} & * & \textit{Beyond Gauss? A more accurate model for LISA astrophysical noise sources.}
\newline{}
Heterogeneous Data and Large Representation Models in Science, Toulouse, France, 2024/10/01.
\vspace{0.05cm}\\
%
\textbf{24.} &  & \textit{LISA stellar-mass black holes informed by the GWTC-3 population: event rates and parameters reconstruction.}
\newline{}
15th International LISA Symposium, Dublin, Ireland, 2024/07/08.
\vspace{0.05cm}\\
%
\textbf{23.} & * & \textit{LISA data analysis: from the stochastic background to the Milky Way.}
\newline{}
11th LISA Cosmology Working Group Workshop, Porto, Portugal, 2024/06/19.
\vspace{0.05cm}\\
%
\textbf{22.} & * & \textit{An introduction to Bayesian Inference.}
\newline{}
International Pulsar Timing Array Student Week, Milan, Italy, 2024/06/17.
\vspace{0.05cm}\\
%
\textbf{21.} & * & \textit{Statistical challenges in LISA data analysis.}
\newline{}
LAUTARO joint meeting, GSSI-University of Milano-Bicocca, Milano, Italy, 2024/04/17.
\vspace{0.05cm}\\
%
\textbf{20.} &  & \textit{From mHz to kHz: stochastic background implications on astrophysical sources and population reconstruction.}
\newline{}
LISA Astrophysics working group workshop, University of Milano-Bicocca, Milano, Italy, 2023/09/13.
\vspace{0.05cm}\\
%
\textbf{19.} &  & \textit{Non-gaussian gravitational wave backgrounds across the GW spectrum.}
\newline{}
XXV Sigrav conference on general relativity and gravitation, SISSA, Trieste, Italy, 2023/09/04.
\vspace{0.05cm}\\
%
\textbf{18.} & * & \textit{LISA SGWB data analysis (session chair).}
\newline{}
Data Analysis Challenges for SGWB Workshop, CERN, Geneva, Switzerland, 2023/07/19.
\vspace{0.05cm}\\
%
\textbf{17.} & * & \textit{Global Fit and foregrounds.}
\newline{}
LISA SGWB detection brainstorming, Univ. of Geneva, Geneva, Switzerland, 2023/07/17.
\vspace{0.05cm}\\
%
\textbf{16.} & * & \textit{Beyond functional forms: non-parametric methods. (panelist talk).}
\newline{}
Gravitational-wave populations: What's next?, University of Milano-Bicocca, Milan, Italy, 2023/07/01.
\vspace{0.05cm}\\
%
\textbf{15.} &  & \textit{The last three years : multiband gravitational-wave observations of stellar-mass binary black holes.}
\newline{}
LISA Astrophysics working group workshop, University of Birmingham, Birmingham, UK, 2022/06/23.
\vspace{0.05cm}\\
%
\textbf{14.} &  & \textit{The last three years : multiband gravitational-wave observations of stellar-mass binary black holes.}
\newline{}
American Physical Society (APS) April meeting, New York (NY), USA, 2022/04/12.
\vspace{0.05cm}\\
%
\textbf{13.} &  & \textit{Bayesian parameter estimation of stellar-mass black-hole binaries with LISA.}
\newline{}
XXIV Sigrav conference on general relativity and gravitation, Urbino, Italy, 2021/09/08.
\vspace{0.05cm}\\
%
\textbf{12.} &  & \textit{Chirp: a web and smartphone application for visualization of gravitational-wave alerts.}
\newline{}
14th Amaldi Conference on Gravitational Waves, (online), 2021/07/21.
\vspace{0.05cm}\\
%
\textbf{11.} & * & \textit{Search for lensing signatures in the gravitational-wave observations from the first half of LIGO-Virgo’s third observing run.}
\newline{}
2nd EPS conference on gravitation, (online, on behalf of LVK), 2021/05/27.
\vspace{0.05cm}\\
%
\textbf{10.} & * & \textit{Bayesian parameter estimation of stellar-mass black-hole binaries with LISA.}
\newline{}
LISA Data Challenge meeting, (online), 2021/06/17.
\vspace{0.05cm}\\
%
\textbf{9.} & * & \textit{Search for lensing signatures in the gravitational-wave observations from the first half of LIGO-Virgo’s third observing run.}
\newline{}
Webinar on behalf of the LVK collaboration, (online), 2021/05/27.
\vspace{0.05cm}\\
%
\textbf{8.} &  & \textit{Milky Way Satellites Shining Bright in Gravitational Waves.}
\newline{}
13th LISA Symposium, (online), 2020/09/13.
\vspace{0.05cm}\\
%
\textbf{7.} &  & \textit{Constraining the Lensing of Binary Black Holes from Their Stochastic Background.}
\newline{}
LISA Sprint workshop, CCA, Flatiron Institute, New York (NY), USA, 2020/03/04.
\vspace{0.05cm}\\
%
\textbf{6.} &  & \textit{Multiple source detection in GW astronomy: the label switching problem.}
\newline{}
30th Texas Symposium, University of Portsmouth, Portsmouth, UK, 2019/12/12.
\vspace{0.05cm}\\
%
\textbf{5.} &  & \textit{Non-gaussian Stochastic background search with importance sampling.}
\newline{}
LIGO, Virgo, KAGRA September meeting, Warsaw, Poland, 2019/09/01.
\vspace{0.05cm}\\
%
\textbf{4.} &  & \textit{An improved detector for non-Gaussian stochastic background.}
\newline{}
Stochastic Background Data Analysis for LISA meeting, Instituto de Fisica Teorica, Madrid, Spain, 2019/06/01.
\vspace{0.05cm}\\
%
\textbf{3.} &  & \textit{Hierarchical nonparametric density estimation for population inference.}
\newline{}
LIGO, Virgo, KAGRA March meeting, Winsconsin, USA, 2019/03/18.
\vspace{0.05cm}\\
%
\textbf{2.} &  & \textit{Fast Evaluation of Campbell processes N–point correlation functions.}
\newline{}
Astro Hack Week: Data Science for Next-Generation Astronomy, Lorentz Center, Leiden, The Netherlands, 2018/08/01.
\vspace{0.05cm}\\
%
\textbf{1.} &  & \textit{Stochastic Gravitational Wave Background Data Analysis for Radler.}
\newline{}
5th LISA Cosmology Working Group workshop, Physicum, University of Helsinki, Helsinki, Finland, 2018/06/01.
\vspace{0.05cm}\\
%
\end{longtable} }
\textcolor{color1}{\textbf{Talks at department seminars:}}
\vspace{-0.5cm}

\cvitem{}{\small\hspace{-1cm}\begin{longtable}{rp{0.3cm}p{15.8cm}}
%
\textbf{10.} & * & \textit{Fast LISA inference using Gaussian processes.}
\newline{}
University of Geneva, Geneva, Switzerland, 2025/05/21.
\vspace{0.05cm}\\
%
\textbf{9.} & * & \textit{Emergence of Milky Way structure in the first year of LISA data.}
\newline{}
Department of Physics, University of Pisa, Pisa, Italy, 2025/05/16.
\vspace{0.05cm}\\
%
\textbf{8.} & * & \textit{Statistical challenges in GW inference: an application of field theory to direct population reconstruction in LISA.}
\newline{}
APP seminar, SISSA, Trieste, Italy, 2024/05/06.
\vspace{0.05cm}\\
%
\textbf{7.} & * & \textit{GRAF: Gravitational waves data and global fit.}
\newline{}
Department of Physics, University of Milano-Bicocca, Milan, Italy, 2023/12/14.
\vspace{0.05cm}\\
%
\textbf{6.} & * & \textit{LISA global inference: statistical and modelling challenges  for the Milky Way.}
\newline{}
Max Planck Institute for Astrophysics, Garching, Germany, 2023/11/29.
\vspace{0.05cm}\\
%
\textbf{5.} & * & \textit{LISA Global inference: modelling, statistical, and computational challenges.}
\newline{}
Department of Physics, University of Pisa, Pisa, Italy, 2023/10/04.
\vspace{0.05cm}\\
%
\textbf{4.} & * & \textit{Gravitational waves in the many sources, many detectors era.}
\newline{}
Institute for Mathematics and Physics, University of Stavanger, Stavanger, Norway, 2022/09/29.
\vspace{0.05cm}\\
%
\textbf{3.} & * & \textit{Stellar mass binary black holes : what, when, and where.}
\newline{}
Astroparticule et cosmologie, Universitè Paris Citè, Paris, France, 2022/06/12, (online).
\vspace{0.05cm}\\
%
\textbf{2.} & * & \textit{The last three years: multiband gravitational-wave observations of stellar-mass binary black holes.}
\newline{}
Physics Department, Columbia University, New York (NY), USA, 2022/04/07.
\vspace{0.05cm}\\
%
\textbf{1.} & * & \textit{Set the alarm : Bayesian parameter estimation of stellar-mass black-hole binaries with LISA.}
\newline{}
Sun Yat-sen University, Zhuhai, China, 2021/07/30, (online).
\vspace{0.05cm}\\
%
\end{longtable} }
\textcolor{color1}{\textbf{Outreach \& public engagement talks:}}
\vspace{-0.5cm}

\cvitem{}{\small\hspace{-1cm}\begin{longtable}{rp{0.3cm}p{15.8cm}}
%
\textbf{5.} &  & \textit{Onde gravitazionali: ascoltare l'Universo anzich'e solo guardarlo.}
\newline{}
University of Milano-Bicocca, Milan, Italy, 2024.
\vspace{0.05cm}\\
%
\textbf{4.} &  & \textit{An orchestra of lasers and gravitational waves.}
\newline{}
Pint of Science 2024, Milan, Italy, 2024.
\vspace{0.05cm}\\
%
\textbf{3.} &  & \textit{Gravitational-waves in space and on Earth.}
\newline{}
Manchester Museum of Science and Industry, Manchester, UK, 2018.
\vspace{0.05cm}\\
%
\textbf{2.} &  & \textit{An orchestra of lasers and gravitational waves.}
\newline{}
Manchester Museum of Science and Industry, Manchester, UK, 2018.
\vspace{0.05cm}\\
%
\textbf{1.} &  & \textit{A Universe of waves.}
\newline{}
Science Caf'e, Italy, 2018.
\vspace{0.05cm}\\
%
\end{longtable} }
%mark_CVshort

\vspace{0.3cm}
\noindent
\begin{minipage}[t]{0.48\textwidth}
	\raggedright
	\textbf{Data:} \today
\end{minipage}
\hfill
\begin{minipage}[t]{0.48\textwidth}
	\raggedleft
	\textbf{Firma:} \rule{5cm}{0.4pt}
\end{minipage}

\end{document}
